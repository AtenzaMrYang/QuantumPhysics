\section{波函数与薛定谔方程}

\subsection{波函数的统计诠释}
微观粒子具有波粒二象性,其波长和频率分别为
$$
    \lambda=\frac{h}{p}, \quad \nu=\frac{E}{h}
$$
微观粒子的运动状态可以用平面波函数来描述
$$
    \psi_{\vec{p}}=A\exp\left[\frac{\mathrm{i}}{\hbar}\left(\vec{p}\cdot\vec{r}-Et\right)\right]
$$
对于概率分布来说,我们更关心相对概率分布
$$
    \left|\frac{C\psi(\vec{r}_1)}{C\psi(\vec{r}_2)}\right|^2 = \left|\frac{\psi(\vec{r}_1)}{\psi(\vec{r}_2)}\right|^2
$$
所以一般的波函数具有常数因子不定性,类似地
$$
    \left|\psi(\vec{r})\mathrm{e}^{\mathrm{i}\alpha}\right|^2
    = |\psi(\vec{r})|^2\mathrm{e}^{\mathrm{i}\alpha}\cdot\mathrm{e}^{-\mathrm{i}\alpha}
    = \left|\psi(\vec{r})\right|^2
$$
一般的波函数也具有位相因子不定性,当然,规范的波函数必须在全空间满足归一化
$$
    \iiint\left|\psi(\vec{r})\right|^2\,\mathrm{d}x\mathrm{d}y\mathrm{d}z=1
$$
为今后表述方便,可以引入符号
$$
    (\psi, \psi)=\int\psi^*\psi\,\mathrm{d}\tau=\int|\psi|^2\,\mathrm{d}\tau
$$
这样,归一化条件就可以简记为
$$
    (\psi, \psi)=1
$$

\subsection{力学量算符}
一维形式的动量算符
$$
    \hat{p}_x = -\mathrm{i}\hbar\frac{\mathrm{d}}{\mathrm{d}x}
$$
动能算符
$$
    \hat{T}=\frac{\hat{p}^2}{2m}
    =\frac{1}{2m}\left(-\mathrm{i}\hbar\nabla\right)^2
    =-\frac{\hbar^2}{2m}\nabla^2
$$
Hamilton算符
$$
    \hat{H}=\hat{T}+\hat{V}(\vec{r})=-\frac{\hbar^2}{2m}\nabla^2+V(\vec{r})
$$
角动量算符
$$
    \hat{\boldsymbol{l}}
    = \boldsymbol{r} \times \hat{\boldsymbol{p}}
    = \boldsymbol{r} \times (-\mathrm{i}\hbar\nabla)
    = \hat{l}_x\vec{e}_x + \hat{l}_y\vec{e}_y + \hat{l}_z\vec{e}_z
$$

\subsection{薛定谔方程}
一般粒子的薛定谔方程是
$$
    \mathrm{i}\hbar\frac{\partial }{\partial t}\psi(\vec{r}, t) = \left[-\frac{\hbar^2}{2m}\nabla^2 + V(\vec{r}, t)\right]\psi(\vec{r}, t)
$$
特殊地,对于势场不随时间变化的\textbf{定态}
$$
    \left[-\frac{\hbar^2}{2m}\nabla^2 + V(\vec{r})\right]\psi(\vec{r}) = E\psi(\vec{r})
$$
能量本征方程为
$$
    \hat{H}\psi(\vec{r}) =E\psi(\vec{r})
$$


\begin{question}{题目1}
    求与下列各粒子相关的de Broglie波波长:
    \begin{enumerate}
        \item[(1)] 能量为100电子伏特的自由电子;
        \item[(2)] 能量为0.1电子伏特的自由中子;
        \item[(3)] 能量为0.1电子伏特、质量为1克的自由粒子;
        \item[(4)] 温度$T=\qty{1}{k}$时,具有动能$E=\dfrac{3kT}{2}$的氦原子,其中$k$为玻尔兹曼常数.
    \end{enumerate}
\end{question}
\begin{solution}
    根据粒子的动能与动量之间的关系
    \begin{equation}\label{能量与动量关系}
        E=\frac{1}{2}mv^2=\frac{p^2}{2m}
    \end{equation}
    结合德布罗意波波长的表达式
    \begin{equation}\label{德布罗意关系}
        \lambda=\frac{h}{p}=\frac{h}{\sqrt{2mE}}
    \end{equation}
    (1) 题设自由电子的德布罗意波波长为
    $$
        \lambda_{e}=\frac{h}{\sqrt{2m_eE_e}}=1.23\times10^{-10}\,{\rm m}
    $$
    (2) 题设自由中子的德布罗意波波长为
    $$
        \lambda_{\rm n}=\frac{h}{\sqrt{2m_{\rm n}E_{\rm n}}}=9.04\times10^{-11}\,{\rm m}
    $$
    (3) 题设自由粒子的德布罗意波波长为
    $$
        \lambda=\frac{h}{\sqrt{2mE}}=1.17\times10^{-22} \,{\rm m}
    $$
    (4) 氦原子包含2个中子和2个质子,其德布罗意波波长为
    $$
        \lambda_{\rm He}=\frac{h}{\sqrt{2m_{\rm He}E_{\rm He}}}
        =\frac{h}{\sqrt{2 \cdot m_{\rm He}\cdot\frac{3}{2}kT}}
        =\frac{h}{\sqrt{3\cdot(4m_{\rm n})\cdot\frac{3}{2}kT}}
        =1.25\times10^{-9} \,{\rm m}
    $$
\end{solution}

\begin{question}{题目2}
    设一电子被电势差$U$所加速,最后打在靶上. 若电子的动能转化为一光子,求当这光子相应的光波波长分别为5000\AA(可见光),1\AA(X射线),0.001\AA($\gamma$射线)时,加速电子所需的电势差各是多少?
\end{question}
\begin{solution}
    电子的动能为
    $$
        E_k=eU
    $$
    光子的能量为
    $$
        E=h\nu=\frac{hc}{\lambda}
    $$
    电子的动能会全部转化为光子的能量,因此电势差
    $$
        U=\frac{hc}{e\lambda}
        =\frac{hc}{e}\cdot\frac{1}{\lambda}
        =\frac{1.24\times10^{-6}}{\lambda}
    $$
    当光子对应的光波波长为 5000\AA 时,电势差为
    $$
        U=\frac{1.24\times10^{-6}}{\lambda}
        =\frac{1.24\times10^{-6}}{5000\times10^{-10}}
        =2.48\,\rm{V}
    $$
    当光子对应的光波波长为 1\AA 时,电势差为
    $$
        U=\frac{1.24\times10^{-6}}{\lambda}
        =\frac{1.24\times10^{-6}}{10^{-10}}
        =1.24\times10^4\,\rm{V}
    $$
    当光子对应的光波波长为 0.001\AA 时,电势差为
    $$
        U=\frac{1.24\times10^{-6}}{\lambda}
        =\frac{1.24\times10^{-6}}{0.001\times10^{-10}}
        =1.24\times10^{7}\,\rm{V}
    $$
\end{solution}

\begin{question}{练习5}
    设用球坐标表示,粒子波函数表为$\psi(\rho, \theta, \varphi)$,求:
    \begin{enumerate}
        \item[(1)] 粒子在球壳$(r, r+\mathrm{d}r)$中被测到的概率;
        \item[(2)] 在$(\theta, \varphi)$方向的立体角元$\mathrm{d}\Omega=\sin\theta\,\mathrm{d}\theta\mathrm{d}\varphi$中找到粒子的概率.
    \end{enumerate}
\end{question}
\begin{solution}
    粒子在$(r\to r+\mathrm{d}r, \theta \to \theta+\mathrm{d}\theta, \varphi \to \varphi+\mathrm{d}\varphi)$范围内被探测到的概率为
    $$
        P = \left|\psi(r, \theta, \varphi)\right|^2r^2\sin\theta \,\mathrm{d}r\mathrm{d}\theta\mathrm{d}\varphi
    $$
    (1) 粒子在球壳$(r, r+\mathrm{d}r)$中被测到的概率为
    $$
        P=\left[\int_{0}^{\pi}\sin\theta\,\mathrm{d}\theta\int_{0}^{2\pi}\left|\psi(r, \theta, \varphi)\right|^2\,\mathrm{d}\varphi\right]r^2\,\mathrm{d}r
    $$
    (2) 在$(\theta, \varphi)$方向的立体角元内找到粒子的概率为
    $$
        P=\left[\int_{0}^{+\infty}\left|\psi(r, \theta, \varphi)\right|^2r^2\,\mathrm{d}r\right]\sin\theta\,\mathrm{d}\theta\mathrm{d}\varphi
    $$
\end{solution}


\begin{question}{习题20}
    设$\varphi(x)=Ax(a-x)$,其中$0 \leqslant x \leqslant a$,求:
    \begin{enumerate}
        \item[(1)] 归一化常数$A$
        \item[(2)] 在何处找到粒子的概率最大?
    \end{enumerate}
\end{question}
\begin{solution}
    按照统计诠释,一个真实的波函数必须满足归一化条件
    \begin{equation}\label{波函数归一化条件}
        \iiint_{-\infty}^{+\infty}|\psi(\vec{r})|^2\,\mathrm{d}x\mathrm{d}y\mathrm{d}z=1
    \end{equation}
    (1) 让波函数在全空间归一化
    $$
        \int_{-\infty}^{+\infty}|\varphi(x)|^2\,\mathrm{d}x
        = \int_0^a |\varphi(x)|^2\,\mathrm{d}x
        = \int_0^a |Ax(a-x)|^2\,\mathrm{d}x
        = |A|^2\left(\frac15x^5-\frac12ax^4+\frac13a^2x^3\right)_0^a
        = 1
    $$
    解得
    $$
        A=\sqrt{\frac{30}{a^5}}
    $$
    (2) 粒子的概率密度为
    $$
        \rho(x) = |Ax(a-x)|^2 = \frac{30}{a^5}x^2(a-x)^2
    $$
    对概率密度求导,寻找极值点
    $$
        \rho'(x)=\frac{30}{a^5}[4x^3-6ax^2+2a^2x]=\frac{60}{a^5}x(2x-a)(x-a)
    $$
    因此,$\rho(x)$在$\left(0,\dfrac{a}{2}\right)$上单调递增,在$\left(\dfrac{a}{2},a\right)$上单调递减,在极大值点$x=\dfrac{a}{2}$附近找到粒子的概率最大.
\end{solution}


\begin{question}{习题21}
    若粒子只在一维空间中运动,它的状态可用波函数
    $$
        \psi(x, t) = \begin{dcases}
            A\sin\frac{\pi x}{a}\mathrm{e}^{-\frac{\mathrm{i}}{h}Et}, & 0 \leqslant x \leqslant a \\
            0,                                                        & \text{其他}                 \\
        \end{dcases}
    $$
    来描述,式中$E$和$a$分别为确定的常数,而$A$是任意常数,求:
    \begin{itemize}
        \item[(1)] 归一化的波函数;
        \item[(2)] 概率密度$w(x, t)$;
        \item[(3)] 在何处找到粒子的概率最大?
        \item[(4)] $\bar{x}$和$\overline{x^2}$的值.
    \end{itemize}
\end{question}
\begin{solution}
    (1) 先将波函数归一化
    $$
        \int_{-\infty}^{+\infty}|\psi(x)|^2\mathrm{d}x
        =\int_0^a\left|A\sin\frac{\pi x}{a}\right|^2\mathrm{d}x
        =\int_0^a\frac{A^2}{2}\left(1-\cos\frac{2\pi x}{a}\right)\mathrm{d}x
        =\int_0^a\frac{A^2}{2}\mathrm{d}x
        =1
    $$
    解得
    $$
        A = \sqrt{\frac{2}{a}}
    $$
    所以波函数为
    $$
        \psi(x, t) = \begin{dcases}
            \sqrt{\frac{2}{a}}\sin\frac{\pi x}{a}\mathrm{e}^{-\frac{\mathrm{i}}{h}Et}, & 0 \leqslant x \leqslant a \\
            0,                                                                         & \text{其他}                 \\
        \end{dcases}
    $$
    (2) 概率密度$w(x,t)$为
    $$
        w(x,t) = \left|\psi(x,t)\right|^2 = \begin{dcases}
            \frac{2}{a}\sin^2\frac{\pi x}{a}, & 0 \leqslant x \leqslant a \\
            0,                                & \text{其他}
        \end{dcases}
    $$
    (3) 对概率密度函数求导,寻找极值点
    $$
        \frac{\partial w(x,t)}{\partial x} = \begin{dcases}
            \frac{2\pi}{a^2}\sin\frac{2\pi x}{a}, & 0\leqslant x\leqslant a \\
            0,                                    & \text{其他}
        \end{dcases}
    $$
    综上,$w(x, t)$在$\left(0,\dfrac{a}{2}\right)$上单调递增,在$\left(\dfrac{a}{2},a\right)$上单调递减,在极大值点$x=\dfrac{a}{2}$附近找到粒子的概率最大.\\
    (4) 概率密度函数为
    $$
        w(x)=\left|\psi(x)\right|^2
        = \frac{2}{a}\sin^2\frac{\pi x}{a}
        = \frac{1}{a}\left(1-\cos\frac{2\pi x}{a}\right)
    $$
    求物理量$g(x)$均值的通用公式是
    $$
        \overline{g(x)} = \int_{-\infty}^{+\infty}w(x)g(x) \,\mathrm{d}x
    $$
    于是分别有
    $$
        \overline{x}
        % =\int_{-\infty}^{+\infty}xw(x)\,\mathrm{d}x
        =\int_0^a\frac{x}{a}\left(1-\cos\frac{2\pi x}{a}\right)\,\mathrm{d}x
        =\left.\left(\frac{x^{2}}{2a}-\frac{a^{2}}{4\pi^{2}}\cos\frac{2\pi x}{a}+\frac{ax}{2\pi}\sin\frac{2\pi x}{a}\right)\right|_{0}^{a}
        =\frac{a}{2}
    $$
    $$%积分的时候令k=2pi/a
        \begin{aligned}
            \overline{x^2}
            %  & =\int_{-\infty}^{+\infty}x^2w(x)\,\mathrm{d}x 
             & =\int_0^a\frac{x^2}{a}\left(1-\cos\frac{2\pi x}{a}\right)\mathrm{d}x
             & =\left.\left(\frac{x^3}{3a}-\frac{x^2}{2\pi}\sin\frac{2\pi x}{a}-\frac{ax}{2\pi^2}\cos\frac{2\pi x}{a}+\frac{a^3}{4\pi^3}\sin\frac{2\pi x}{a}\right)\right|_0^a
             & = \frac{a^2}{3} - \frac{a^2}{2\pi^2}
        \end{aligned}
    $$
\end{solution}



\begin{question}{题目1.3}
    对于一维自由粒子:
    \begin{enumerate}
        \item[(a)] 设波函数为$\displaystyle \psi_p(x)=\frac{1}{\sqrt{2\pi\hbar}}\mathrm{e}^{\mathrm{i}px/\hbar}$,试用Hamilton算符$\displaystyle \hat{H}=\frac{\hat{p}^2}{2m}=-\frac{\hbar^2}{2m}\frac{\mathrm{d}^2}{\mathrm{d}x^2}$对$\psi_p(x)$运算,验证$\displaystyle \hat{H}\psi_p(x)=\frac{p^2}{2m}\psi_p(x)$. 说明动量本征态$\psi_p(x)$也是Hamliton量(能量)本征态,本征值为$\displaystyle E=\cfrac{p^2}{2m}$.
        \item[(b)] 设粒子在初始($t=0$)时刻$\psi(x,0)=\psi_p(x)$,求$\psi(x,t)$.
        \item[(c)] 设波函数为$\displaystyle \psi(x)=\delta(x)=\frac{1}{2\pi}\int\mathrm{e}^{\mathrm{i}kx}\,\mathrm{d}k=\frac{1}{2\pi}\int\mathrm{e}^{\mathrm{i}px/\hbar}\,\mathrm{d}p$,可以看成是无穷多个平面波$\mathrm{e}^{\mathrm{i}px}$的叠加,即无穷多个动量本征态$\mathrm{e}^{\mathrm{i}px}$的叠加.试问$\psi(x)=\delta(x)$是否是能量本征态?
        \item[(d)] 设粒子在$t=0$时刻$\psi(x,0)=\delta(x)$,求$\psi(x,t)$.
    \end{enumerate}
\end{question}
\begin{solution}
    对于一维自由粒子:
    \paragraph{(a)} 我们将Hamilton算符作用于波函数
    $$
        \hat{H}\psi_p(x)
        =-\frac{\hbar}{2m}\frac{\mathrm{d}^2}{\mathrm{d}x^2}\left(\frac{1}{\sqrt{2\pi\hbar}}\mathrm{e}^{\mathrm{i}px/\hbar}\right)
        =-\frac{\hbar^2}{2m}\left(\frac{\mathrm{i}p}{\hbar}\right)^2 \frac{1}{\sqrt{2\pi\hbar}}\mathrm{e}^{\mathrm{i}px/\hbar}
        =\frac{p^2}{2m}\psi_p(x)
    $$
    这说明动量本征态$\psi_p(x)$也是Hamilton量的本征态,且本征值为$E=\dfrac{p^2}{2m}$

    \paragraph{(b)}粒子在初始时刻$\psi(x,0) = \psi_p(x) = \mathrm{e}^{\mathrm{i}p_0x/\hbar}$,结合Fourier变换公式$\eqref{复数形式的傅里叶变换}$
    $$
        \phi(p)=\mathcal{F}[\psi(x)]
        =\frac{1}{\sqrt{2\pi\hbar}}\int_{-\infty}^{+\infty}\mathrm{e}^{\mathrm{i}p_0x/\hbar}\mathrm{e}^{-\mathrm{i}px/\hbar}\,\mathrm{d}x
        =\sqrt{2\pi\hbar}\delta(p-p_0)
    $$
    由于$\psi(x,0)$是能量本征态
    $$
        \psi(x, t) = \frac{1}{\sqrt{2\pi\hbar}}\int_{-\infty}^{+\infty}\sqrt{2\pi\hbar}\mathrm{e}^{\mathrm{i}px/\hbar}\delta(p-p_0)\,\mathrm{d}p\cdot\mathrm{e}^{-\mathrm{i}Et/\hbar}
        =\mathrm{e}^{-\mathrm{i}p_0x/\hbar-\mathrm{i}Et/\hbar}
    $$

    \paragraph{(c)} 对于自由粒子而言,动量本征态和能量本征态是等价的,但是题目中的$\psi(x, 0)=\delta(x)$是无穷多个动量本征态$\mathrm{e}^{\mathrm{i}px}$的叠加(即所谓叠加态),所以显然不是能量本征态。
\end{solution}

\begin{question}{习题12}
    由下列两个定态波函数计算几率流密度:
    \begin{enumerate}
        \item[(1)] $\psi_1(r)=\dfrac{A}{r}\mathrm{e}^{\mathrm{i}kr}$
        \item[(2)] $\psi_2(r)=\dfrac{A}{r}\mathrm{e}^{-\mathrm{i}kr}$
    \end{enumerate}
    从所得结果证明$\psi_1(r)$表示向外传播的球面波,$\psi_2(r)$表示向内(即向原点)传播的平面波.
\end{question}
\begin{solution}
    概率流密度的表达式为\footnote{球坐标系中的$\nabla$算符为$$\nabla=\frac{\partial }{\partial r}\vec{e}_r + \frac{1}{r}\frac{\partial }{\partial \theta}\vec{e}_{\theta} + \frac{1}{r\sin\theta}\frac{\partial }{\partial \varphi}\vec{e}_{\varphi}$$}
    \begin{equation}\label{概率流密度}
        \vec{j}(\vec{r}, t) = -\frac{\mathrm{i}\hbar}{2m}\left(\psi^*\nabla\psi-\psi\nabla\psi^*\right)
    \end{equation}
    分别计算$\psi_1(x)$和$\psi_2(x)$的概率流密度
    $$
        \vec{j}_1(\vec{r}) = -\frac{\mathrm{i}\hbar}{2m}\left\{\left(\frac{A}{r}\mathrm{e}^{-\mathrm{i}kr}\cdot\frac{A}{r}\mathrm{e}^{\mathrm{i}kr}\mathrm{i}k\right) - \left[\frac{A}{r}\mathrm{e}^{\mathrm{i}kr}\cdot\frac{A}{r}\mathrm{e}^{-\mathrm{i}kr}(-\mathrm{i}k)\right]\right\}\vec{e}_r
        =\frac{\hbar}{m}\frac{A^2k}{r^2}\vec{e}_r
    $$
    $$
        \vec{j}_2(\vec{r}) = -\frac{\mathrm{i}\hbar}{2m}\left\{\left[\frac{A}{r}\mathrm{e}^{\mathrm{i}kr}\cdot\frac{A}{r}\mathrm{e}^{-\mathrm{i}kr}(-\mathrm{i}k)\right]-\left(\frac{A}{r}\mathrm{e}^{-\mathrm{i}kr}\cdot\frac{A}{r}\mathrm{e}^{\mathrm{i}kr}\mathrm{i}k\right)\right\}\vec{e}_r
        =-\frac{\hbar}{m}\frac{A^2k}{r^2}\vec{e}_r
    $$
    由此可见,$\psi_1(r)$表示向外传播的球面波,$\psi_2(r)$表示向内(即向原点)传播的平面波.
\end{solution}
