\section{一维势场中的粒子}
\begin{question}{P32练习}
    试取无限深方势阱的中心为坐标原点,即
    $$
        V(x) = \begin{dcases}
            0,      & |x|<\frac{a}{2}          \\
            \infty, & |x|\geqslant \frac{a}{2} \\
        \end{dcases}
    $$
    证明粒子的能量仍为
    $$
        E = E_n = \frac{\hbar^2\pi^2n^2}{2ma^2}, \quad n=1,2,3,\cdots
    $$
    但波函数表示为
    $$
        \psi_n(x) = \begin{cases}
            \sqrt{\dfrac{a}{2}}\cos\left(\dfrac{n\pi x}{a}\right), n=1,3,5,\cdots, & |x|\leqslant\dfrac{a}{2} \\
            \sqrt{\dfrac{a}{2}}\sin\left(\dfrac{n\pi x}{a}\right), n=2,4,6,\cdots, & |x|\leqslant\dfrac{a}{2} \\
            0,                                                                     & |x|\geqslant\dfrac{a}{2} \\
        \end{cases}
    $$
\end{question}
\begin{solution}
\end{solution}




\begin{question}{习题2.1}
    设粒子限制在矩形匣子中运动,即
    $$
        V(x) = \begin{cases}
            0,      & 0<x<a, 0<y<b, 0<z<c \\
            \infty, & \text{else}         \\
        \end{cases}
    $$
    求粒子的能量本征值和本征波函数. 如$a=b=c$,讨论能级的简并度.
\end{question}
\begin{solution}
\end{solution}




\begin{question}{习题2.2}
    设粒子处于一维无限深方势阱中,
    $$
        V(x) = \begin{cases}
            0,      & 0<x<a    \\
            \infty, & x<0, x>a \\
        \end{cases}
    $$
    证明处于能量本征态$\psi_n(x)$的粒子,$\overline{x}=\cfrac{a}{2}$
    $$
        \overline{\left(x-\overline{x}\right)^2} = \frac{a^2}{12}\left(1-\frac{6}{n^2\pi^2}\right)
    $$
    讨论$n\to\infty$的情况,并于经典力学计算结果比较.
\end{question}
\begin{solution}
\end{solution}



