\section{微扰论}

\begin{question}{Ex85}
    设哈密顿量在能量表象中的矩阵形式为
    $$
        H=\begin{pmatrix}
            1 & c & 0   \\
            c & 3 & 0   \\
            0 & 0 & c-1
        \end{pmatrix}
    $$
    \begin{enumerate}
        \item 求 $H$ 的精确本征值;
        \item 设$c \ll 1$,利用微扰法理论求能量至二级修正;
        \item 在什么条件下上述两种结果一致?
    \end{enumerate}
\end{question}


\begin{solution}
    % 本征值问题可以表示为 $H\psi = E\psi$,其中 $H$ 是给定的哈密顿量矩阵,$\psi$ 是对应的本征向量,$E$ 是本征值。
    (1) 解本征值问题等价于求解特征方程 $\det(H-EI)=0$,其中 $I$ 是单位矩阵。代入给定的哈密顿量矩阵 $H$,我们得到:
    $$
        \begin{vmatrix}
            1-E & c   & 0     \\
            c   & 3-E & 0     \\
            0   & 0   & c-1-E
        \end{vmatrix} = 0
    $$
    计算行列式后化简得到:
    $$
        (1-E)[(3-E)(c-1-E) - 0] - c[c(c-1-E)] = 0
    $$
    展开化简后得到:
    $$
        (1-E)(3c - 3E - cE + E^2 - cE + c^2) = 0
    $$
    化简得到:
    $$
        (1-E)(E^2 - (3+2c)E + 3c - c^2) = 0
    $$
    解这个二次方程可以得到三个本征值
    $$
        E_1, \quad E_2, \quad E_3
    $$
    (2) 对于微扰参数 $c \ll 1$,我们将哈密顿量表示为 $H = H^{(0)} + cH^{(1)}$,其中 $H^{(0)}$ 是未受微扰的哈密顿量,$H^{(1)}$ 是微扰项。
    \begin{itemize}
        \item 在零级近似下,我们使用未受微扰的哈密顿量 $H^{(0)}$ 的本征值 $E^{(0)}$ 和本征态 $\psi^{(0)}$;
        \item 在一级近似下,我们使用微扰项 $H^{(1)}$ 对本征值进行修正。能量修正为:
              $$
                  \Delta E^{(1)}_n = \langle \psi^{(0)}_n | H^{(1)} | \psi^{(0)}_n \rangle
              $$
        \item  在二级近似下,能量修正为:
              $$
                  \Delta E^{(2)}_n = \sum_{m \neq n} \frac{|\langle \psi^{(0)}_m | H^{(1)} | \psi^{(0)}_n \rangle|^2}{E^{(0)}_n - E^{(0)}_m}
              $$
    \end{itemize}
    在一级近似下,我们需要计算能量的修正项 $\Delta E_n^{(1)}$。根据微扰理论,我们有:
    $$
        \Delta E_n^{(1)} = \langle\psi_n^{(0)}|H^{(1)}|\psi_n^{(0)}\rangle
    $$
    代入给定的哈密顿量矩阵形式,我们可以计算一级修正项为:
    $$
        \Delta E_n^{(1)} = \langle \psi_n^{(0)} | cH^{(1)} | \psi_n^{(0)} \rangle
    $$
    根据矩阵乘法的定义,我们可以将上式展开为:
    $$
        \Delta E_n^{(1)} = c \sum_{i,j} \psi_n^{(0)*}(i) H^{(1)}(i,j) \psi_n^{(0)}(j)
    $$
    % 其中,$\psi_n^{(0)}(i)$ 和 $\psi_n^{(0)*}(i)$ 分别表示本征态 $\psi_n^{(0)}$ 的第 $i$ 个分量和其复共轭。\\
    (3) 在二级近似下,我们需要计算能量的二级修正项 $\Delta E_n^{(2)}$。根据微扰理论,我们有:
    $$
        \Delta E_n^{(2)} = \sum_{m \neq n} \frac{|\langle \psi_m^{(0)} | H^{(1)} | \psi_n^{(0)} \rangle|^2}{E_n^{(0)} - E_m^{(0)}}
    $$
    代入给定的哈密顿量矩阵形式,我们可以计算二级修正项为:
    $$
        \Delta E_n^{(2)} = \sum_{m \neq n} \frac{|c \psi_m^{(0)*}(i) H^{(1)}(i,j) \psi_n^{(0)}(j)|^2}{E_n^{(0)} - E_m^{(0)}}
    $$
    根据微扰理论的前提,微扰项应当是一个较小的修正。因此,当微扰参数 $c$ 足够小的时候,即 $c \ll 1$,才能保证一级和二级修正项都是可靠的近似。
\end{solution}



\begin{question}{Ex86}
    设哈密顿量在能量表象中的矩阵形式为
    $$
        H=\begin{pmatrix}
            A+B & A-B \\
            A-B & A+B \\
        \end{pmatrix}
    $$
    其中$A$、$B$为实数,求:
    \begin{enumerate}
        \item 若$A+B \gg A-B$,用微扰法求能量至一级修正;
        \item 直接求能量本征值并和1所得结果进行比较。
    \end{enumerate}
\end{question}
\begin{solution}
    (1) 根据微扰理论,我们将哈密顿量表示为 $H = H^{(0)} + \lambda H^{(1)}$,在零级近似下,我们使用未受微扰的哈密顿量 $H^{(0)}$ 的本征值 $E^{(0)}$ 和本征态 $\psi^{(0)}$;在一级近似下,能量修正为:
    $$
        \Delta E^{(1)}_n = \langle \psi^{(0)}_n | H^{(1)} | \psi^{(0)}_n \rangle
    $$
    对于给定的哈密顿量矩阵,我们可以将其分解为 $H = H^{(0)} + \lambda H^{(1)}$,其中:
    $$
        H^{(0)} = \begin{pmatrix}
            A & A \\
            A & A \\
        \end{pmatrix}
        \quad
        H^{(1)} = \begin{pmatrix}
            B  & -B \\
            -B & B  \\
        \end{pmatrix}
    $$
    在零级近似下,本征值为 $E^{(0)}_1 = E^{(0)}_2 = A$,对应的本征态分别为
    $$
        \psi^{(0)}_1 = \begin{pmatrix} 1 \\ 1 \end{pmatrix}
        \quad
        \psi^{(0)}_2 = \begin{pmatrix} -1 \\ 1 \end{pmatrix}
    $$
    在一级近似下,我们计算能量的修正项 $\Delta E^{(1)}_n$,其中 $n = 1,2$:
    $$
        \Delta E^{(1)}_1
        = \langle \psi^{(0)}_1 | H^{(1)} | \psi^{(0)}_1 \rangle
        = \begin{pmatrix}
            1 & 1
        \end{pmatrix}\begin{pmatrix}
            B  & -B \\
            -B & B
        \end{pmatrix}\begin{pmatrix}
            1 \\
            1
        \end{pmatrix}
        = 2B
    $$
    $$
        \Delta E^{(1)}_2
        = \langle \psi^{(0)}_2 | H^{(1)} | \psi^{(0)}_2 \rangle
        = \begin{pmatrix}
            -1 & 1
        \end{pmatrix}\begin{pmatrix}
            B  & -B \\
            -B & B
        \end{pmatrix}\begin{pmatrix}
            -1 \\ 1
        \end{pmatrix}
        = -2B
    $$
    因此,能量修正项为
    $$
        \Delta E^{(1)}_1 = 2B \quad \Delta E^{(1)}_2 = -2B
    $$
    (2) 设对角化矩阵为$S$,即
    $$
        S^{-1} H S = \begin{pmatrix}
            E_1^{(0)} & 0         \\
            0         & E_2^{(0)} \\
        \end{pmatrix}
    $$
    将哈密顿量矩阵$H$代入上式,我们可以得到
    $$
        S^{-1} H S = \begin{pmatrix}
            A+B & A-B \\
            A-B & A+B \\
        \end{pmatrix} = \begin{pmatrix}
            E_1^{(0)} & 0         \\
            0         & E_2^{(0)} \\
        \end{pmatrix}
    $$
    由此可知
    $$
        S = \begin{pmatrix}
            a & b \\
            c & d \\
        \end{pmatrix}
    $$
    将$S$矩阵代入上式,我们可以得到以下方程组
    $$
        \begin{cases}
            (A+B)a + (A-B)c = E_1^{(0)}a \\
            (A+B)b + (A-B)d = E_1^{(0)}b \\
            (A+B)a - (A-B)c = E_2^{(0)}c \\
            (A+B)b - (A-B)d = E_2^{(0)}d \\
        \end{cases}
    $$
    由于题目中给定的条件是$A+B \gg A-B$,我们可以做近似处理。在这种情况下,等式组的解可以近似为:
    $$
        \begin{cases}
            a \approx 1 \\
            b \approx 0 \\
            c \approx 0 \\
            d \approx 1 \\
        \end{cases}
    $$
    因此,本征态矩阵$S$近似为:
    $$
        S \approx \begin{pmatrix}
            1 & 0 \\
            0 & 1 \\
        \end{pmatrix}
    $$
    接下来,我们将微扰哈密顿量$H'$写成本征态的形式:
    $$
        H' = S^{-1} H S - \begin{pmatrix}
            E_1^{(0)} & 0         \\
            0         & E_2^{(0)} \\
        \end{pmatrix}
    $$
\end{solution}


\begin{question}{Ex75}
    转动惯量为$I$,电矩为$\vec{D}$的平面转子处在均匀电场$\varepsilon$中,电场是在转子运动的平面上,用微扰法求转子能量的修正值。
\end{question}
\begin{solution}
    在这个问题中,转子的零级近似哈密顿量是转子的动能和势能之和:
    $$
        H^{(0)} = \frac{1}{2I}L^2
    $$
    其中 $L$ 是转子的角动量算符,$I$ 是转子的转动惯量。微扰项由转子的电矩 $\vec{D}$ 和均匀电场 $\varepsilon$ 组成:
    $$
        H' = -\vec{D} \cdot \varepsilon
    $$
    能量的一级修正可以表示为:
    $$
        E^{(1)} = \langle \psi^{(0)} | H' | \psi^{(0)} \rangle
    $$
    其中 $|\psi^{(0)}\rangle$ 是转子的零级近似能量本征态。

    转子的零级近似能量本征态是转子的角动量本征态,即 $L^2$ 的本征态。因此,我们可以将 $|\psi^{(0)}\rangle$ 表示为 $|l, m\rangle$,其中 $l$ 是角动量量子数,$m$ 是角动量在 $z$ 方向上的投影量量子数。

    现在,我们将 $H'$ 展开为 $H' = -D\varepsilon \cos \theta$,其中 $\theta$ 是转子运动平面与电场方向的夹角,$D = |\vec{D}|$ 是电矩的大小。

    代入上述表达式,我们可以计算能量的一级修正 $E^{(1)}$:
    $$
        E^{(1)} = \langle l, m | -D\varepsilon \cos \theta | l, m \rangle
    $$

    由于 $|l, m\rangle$ 是角动量的本征态,$\cos \theta$ 是常数,我们可以将 $\cos \theta$ 移至左侧,得到:
    $$
        E^{(1)} = -D\varepsilon \cos \theta \langle l, m | l, m \rangle
    $$

    由于 $|l, m\rangle$ 是正交归一的,即 $\langle l, m | l, m \rangle = 1$,因此能量的一级修正简化为:
    $$
        E^{(1)} = -D\varepsilon \cos \theta
    $$

    这是转子能量的一级修正值。注意,这里的修正值是一个常数,不依赖于角动量量子数 $l$ 和 $m$。
\end{solution}



\begin{question}{Ex76}
    转动惯量为$I$,电矩为$\vec{D}$的平面转子处在均匀弱电场$\varepsilon$中,电场是在转子运动的平面上,用微扰法求转子能量的修正值。
\end{question}
\begin{solution}
    在这个问题中,转子的零级近似哈密顿量是转子的转动能量:
    $$
        H^{(0)} = \frac{L^2}{2I}
    $$
    微扰项由转子的电矩 $\vec{D}$ 和均匀弱电场 $\varepsilon$ 组成:
    $$
        H' = -\vec{D} \cdot \vec{\varepsilon}
    $$
    能量的一级修正可以表示为:
    $$
        E^{(1)} = \langle \psi^{(0)} | H' | \psi^{(0)} \rangle
    $$
    现在,我们将 $H'$ 展开为 $H' = -D \varepsilon \cos \theta$,其中 $\theta$ 是电场矢量和转子角动量矢量的夹角,$D = |\vec{D}|$ 是电矩的大小。代入上述表达式,我们可以计算能量的一级修正 $E^{(1)}$:
    $$
        E^{(1)} = \langle l, m | -D \varepsilon \cos \theta | l, m \rangle
    $$
    由于 $|l, m\rangle$ 是角动量的本征态,$\cos\theta$ 是常数,我们可以将 $\cos\theta$ 移至左侧,得到:
    $$
        E^{(1)} = -D \varepsilon \cos \theta \langle l, m | l, m \rangle
    $$
    由于 $|l, m\rangle$ 是归一化的本征态,即 $\langle l, m | l, m \rangle = 1$,因此能量的一级修正简化为:
    $$
        E^{(1)} = -D \varepsilon \cos \theta
    $$
    这是转子能量的一级修正值。修正值与角动量量子数 $l$ 和 $m$ 无关,仅依赖于电矩的大小 $D$、电场的大小 $\varepsilon$ 和电场方向与转子运动平面的夹角 $\theta$。
\end{solution}
