\section{微扰论}

\subsection{非简并态情形}
一级能级修正
$$
    E_n^{(1)}=H_{nn}'
$$
一级波函数修正
$$
    \psi_n^{(1)}\sum_{m \neq n}\frac{H_{mn}'}{E_n^{(0)}-E_m^{(0)}}\psi_m^{(0)}
$$
一级能级修正

\begin{question}{Ex85}
    设哈密顿量在能量表象中的矩阵形式为
    $$
        H=\begin{pmatrix}
            1 & c & 0   \\
            c & 3 & 0   \\
            0 & 0 & c-1
        \end{pmatrix}
    $$
    \begin{enumerate}
        \item 求 $H$ 的精确本征值;
        \item 设$c \ll 1$,利用微扰法理论求能量至二级修正;
        \item 在什么条件下上述两种结果一致?
    \end{enumerate}
\end{question}


\begin{solution}
    % 本征值问题可以表示为 $H\psi = E\psi$,其中 $H$ 是给定的哈密顿量矩阵,$\psi$ 是对应的本征向量,$E$ 是本征值。
    (1) 解本征值问题等价于求解特征方程 $\det(H-EI)=0$,其中 $I$ 是单位矩阵。代入给定的哈密顿量矩阵 $H$,我们得到:
    $$
        \begin{vmatrix}
            1-E & c   & 0     \\
            c   & 3-E & 0     \\
            0   & 0   & c-1-E
        \end{vmatrix} = 0
    $$
\end{solution}


\begin{question}{Ex75}
    转动惯量为$I$,电矩为$\vec{D}$的平面转子处在均匀电场$\varepsilon$中,电场是在转子运动的平面上,用微扰法求转子能量的修正值。
\end{question}
\begin{solution}
    在这个问题中,转子的零级近似哈密顿量是转子的动能和势能之和:
    $$
        H^{(0)} = \frac{1}{2I}L^2
    $$
    其中 $L$ 是转子的角动量算符,$I$ 是转子的转动惯量。微扰项由转子的电矩 $\vec{D}$ 和均匀电场 $\varepsilon$ 组成:
    $$
        H' = -\vec{D} \cdot \varepsilon
    $$
    能量的一级修正可以表示为:
    $$
        E^{(1)} = \langle \psi^{(0)} | H' | \psi^{(0)} \rangle
    $$
    其中 $|\psi^{(0)}\rangle$ 是转子的零级近似能量本征态。
\end{solution}



\begin{question}{Ex76}
    转动惯量为$I$,电矩为$\vec{D}$的平面转子处在均匀弱电场$\varepsilon$中,电场是在转子运动的平面上,用微扰法求转子能量的修正值。
\end{question}
\begin{solution}
\end{solution}
