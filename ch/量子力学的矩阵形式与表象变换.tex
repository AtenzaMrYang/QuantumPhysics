\section{量子力学的矩阵形式与表象变换}

\begin{question}{题目7.3}
    设一维粒子Hamilton量为$H=\dfrac{p^2}{2m}+V(x)$. 写出 $x$ 表象中$x,p,H$的矩阵元.
\end{question}
\begin{solution}
    坐标$x$在坐标表象中表示为
    $$
        \langle x'|x|x'' \rangle = x'\delta(x'-x'')
    $$
    势能$V(x)$在坐标表象中表示为
    $$
        \langle x'|V(x)|x'' \rangle = V(x')\delta(x'-x'')
    $$
    由此可得各种矩阵元
    $$
        (x)_{x'x''} = \langle x'|x|x'' \rangle
        = \int\delta(x-x')x\delta(x-x'')\,\mathrm{d}x
        = x'\delta(x'-x'')
    $$
    $$
        \begin{aligned}
            (p)_{x'x''}
             & = \langle x'|p|x'' \rangle = \int\delta(x-x')\left(-\mathrm{i}\hbar\frac{\partial }{\partial x}\right)\delta(x-x'')\,\mathrm{d}x \\
             & = -\mathrm{i}\hbar\frac{\partial }{\partial x'}\int\delta(x-x')\delta(x-x'')\,\mathrm{d}x                                        \\
             & = -\mathrm{i}\hbar\frac{\partial }{\partial x'}\delta(x'-x'')
        \end{aligned}
    $$
    $$
        \begin{aligned}
            (H)_{x'x''}
             & =\langle x'|H|x'' \rangle =\int\delta(x-x')H\delta(x-x'')\,\mathrm{d}x                                          \\
             & =\int\delta(x-x')\left[-\frac{\hbar^2}{2m}\frac{\partial^2}{\partial x^2}+V(x)\right]\delta(x-x'')\,\mathrm{d}x \\
             & =-\frac{\hbar^2}{2m}\frac{\partial^2}{\partial x'^2}\delta(x'-x'') + V(x')\delta(x'-x'')                        \\
        \end{aligned}
    $$
\end{solution}


\begin{question}{习题集Ex94}
    已知在$\sigma^2-\sigma_z$表象中,算符$\hat{\sigma}_x$ 和 $\hat{\sigma}_y$的矩阵形式为
    $$
        \sigma_x=\begin{pmatrix}
            0 & 1 \\
            1 & 0
        \end{pmatrix}
        \quad
        \sigma_y=\begin{pmatrix}
            0          & -\mathrm{i} \\
            \mathrm{i} & 0
        \end{pmatrix}
    $$
    \begin{enumerate}
        \item 求它们的本征值与本征函数.
        \item 写出在$\sigma^2-\sigma_z$表象中,算符$\hat{\sigma}_x$的矩阵形式及其本征函数形式.
    \end{enumerate}
\end{question}
\begin{solution}
    (1) 对于算符$\hat{\sigma}_x$,其本征方程$\hat{\sigma}_x|\psi\rangle = \lambda|\psi\rangle$的矩阵形式为
    $$
        \begin{pmatrix}
            0 & 1 \\
            1 & 0
        \end{pmatrix}\begin{pmatrix}
            a \\
            b
        \end{pmatrix}
        =\lambda\begin{pmatrix}
            a \\
            b
        \end{pmatrix}
    $$
    也即方程组
    $$
        b = \lambda a, \quad
        a =\lambda b
    $$
    解得
    $$
        \lambda_1=1, \quad |\psi_1\rangle=\begin{pmatrix}
            1 \\ 1
        \end{pmatrix}
    $$
    $$
        \lambda_2=-1, \quad |\psi_2\rangle=\begin{pmatrix}
            1 \\ -1
        \end{pmatrix}
    $$
    类似地,算符$\hat{\sigma}_y$的本征值和对应的本征函数为
    $$
        \lambda_1=1, \quad |\psi_1\rangle=\begin{pmatrix}
            1 \\ -\mathrm{i}
        \end{pmatrix}
    $$
    $$
        \lambda_2=-1, \quad |\psi_2\rangle=\begin{pmatrix}
            -1 \\ \mathrm{i}
        \end{pmatrix}
    $$
    (2) 在$\sigma^2-\sigma_z$表象中,算符$\hat{\sigma}_x$的矩阵形式可以通过变换矩阵$U$得到
    $$
        U=\frac{1}{\sqrt{2}}\begin{pmatrix}
            1 & 1  \\
            1 & -1
        \end{pmatrix}
    $$
    算符$\hat{\sigma}_x$在$\sigma^2-\sigma_z$表象中的矩阵形式
    $$
        \hat{\sigma}_x' = U\sigma_xU^{-1} = \begin{pmatrix}
            1 & 0 \\
            0 & 1
        \end{pmatrix}
    $$
    而$\hat{\sigma}_x$在此表象中的本征函数为
    $$
        |\psi_1\rangle=\begin{pmatrix}
            1 \\
            0
        \end{pmatrix}
        \quad
        |\psi_2\rangle=\begin{pmatrix}
            0 \\
            1
        \end{pmatrix}
    $$
\end{solution}











