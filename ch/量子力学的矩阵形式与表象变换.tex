\section{量子力学的矩阵形式与表象变换}

\begin{question}{题目7.3}
    设一维粒子Hamilton量为$H=\dfrac{p^2}{2m}+V(x)$. 写出 $x$ 表象中$x,p,H$的矩阵元.
\end{question}
\begin{solution}
    坐标$x$在坐标表象中表示为
    $$
        \langle x'|x|x'' \rangle = x'\delta(x'-x'')
    $$
    势能$V(x)$在坐标表象中表示为
    $$
        \langle x'|V(x)|x'' \rangle = V(x')\delta(x'-x'')
    $$
    由此可得各种矩阵元
    $$
        (x)_{x'x''} = \langle x'|x|x'' \rangle
        = \int\delta(x-x')x\delta(x-x'')\,\mathrm{d}x
        = x'\delta(x'-x'')
    $$
    $$
        \begin{aligned}
            (p)_{x'x''}
             & = \langle x'|p|x'' \rangle = \int\delta(x-x')\left(-\mathrm{i}\hbar\frac{\partial }{\partial x}\right)\delta(x-x'')\,\mathrm{d}x \\
             & = -\mathrm{i}\hbar\frac{\partial }{\partial x'}\int\delta(x-x')\delta(x-x'')\,\mathrm{d}x                                        \\
             & = -\mathrm{i}\hbar\frac{\partial }{\partial x'}\delta(x'-x'')
        \end{aligned}
    $$
    $$
        \begin{aligned}
            (H)_{x'x''}
             & =\langle x'|H|x'' \rangle =\int\delta(x-x')H\delta(x-x'')\,\mathrm{d}x                                          \\
             & =\int\delta(x-x')\left[-\frac{\hbar^2}{2m}\frac{\partial^2}{\partial x^2}+V(x)\right]\delta(x-x'')\,\mathrm{d}x \\
             & =-\frac{\hbar^2}{2m}\frac{\partial^2}{\partial x'^2}\delta(x'-x'') + V(x')\delta(x'-x'')                        \\
        \end{aligned}
    $$
\end{solution}


\begin{question}{习题集Ex94}
    已知在$\sigma^2-\sigma_z$表象中,算符$\hat{\sigma}_x$ 和 $\hat{\sigma}_y$的矩阵形式为
    $$
        \sigma_x=\begin{pmatrix}
            0 & 1 \\
            1 & 0
        \end{pmatrix}
        \quad
        \sigma_y=\begin{pmatrix}
            0          & -\mathrm{i} \\
            \mathrm{i} & 0
        \end{pmatrix}
    $$
    \begin{enumerate}
        \item 求它们的本征值与本征函数.
        \item 写出在$\sigma^2-\sigma_z$表象中,算符$\hat{\sigma}_x$的矩阵形式及其本征函数形式.
    \end{enumerate}
\end{question}
\begin{solution}
    (1) 对于算符$\hat{\sigma}_x$,其本征方程$\hat{\sigma}_x|\psi\rangle = \lambda|\psi\rangle$的矩阵形式为
    $$
        \begin{pmatrix}
            0 & 1 \\
            1 & 0
        \end{pmatrix}\begin{pmatrix}
            a \\
            b
        \end{pmatrix}
        =\lambda\begin{pmatrix}
            a \\
            b
        \end{pmatrix}
    $$
    也即方程组
    $$
        b = \lambda a, \quad
        a =\lambda b
    $$
    解得
    $$
        \lambda_1=1, \quad |\psi_1\rangle=\begin{pmatrix}
            1 \\ 1
        \end{pmatrix}
    $$
    $$
        \lambda_2=-1, \quad |\psi_2\rangle=\begin{pmatrix}
            1 \\ -1
        \end{pmatrix}
    $$
    类似地,算符$\hat{\sigma}_y$的本征值和对应的本征函数为
    $$
        \lambda_1=1, \quad |\psi_1\rangle=\begin{pmatrix}
            1 \\ -\mathrm{i}
        \end{pmatrix}
    $$
    $$
        \lambda_2=-1, \quad |\psi_2\rangle=\begin{pmatrix}
            -1 \\ \mathrm{i}
        \end{pmatrix}
    $$
    (2) 在$\sigma^2-\sigma_z$表象中,算符$\hat{\sigma}_x$的矩阵形式可以通过变换矩阵$U$得到
    $$
        U=\frac{1}{\sqrt{2}}\begin{pmatrix}
            1 & 1  \\
            1 & -1
        \end{pmatrix}
    $$
    算符$\hat{\sigma}_x$在$\sigma^2-\sigma_z$表象中的矩阵形式
    $$
        \hat{\sigma}_x' = U\sigma_xU^{-1} = \begin{pmatrix}
            1 & 0 \\
            0 & 1
        \end{pmatrix}
    $$
    而$\hat{\sigma}_x$在此表象中的本征函数为
    $$
        |\psi_1\rangle=\begin{pmatrix}
            1 \\
            0
        \end{pmatrix}
        \quad
        |\psi_2\rangle=\begin{pmatrix}
            0 \\
            1
        \end{pmatrix}
    $$
\end{solution}



\begin{question}{题目8.1}
    \begin{enumerate}
        \item 在$\sigma_z$表象中,求$\sigma_x$的本征态;
        \item 求$\sigma_z$表象变换到$\sigma_x$表象的变换矩阵;
        \item 验证
              $$
                  S\sigma_xS^{-1} = S\begin{pmatrix} 0 & 1 \\ 1 & 0 \end{pmatrix}S^{-1}=\begin{pmatrix} 1 & 0 \\ 0 & -1 \end{pmatrix}
              $$
    \end{enumerate}
\end{question}
\begin{solution}

\end{solution}



\begin{question}{题目8.3}
    在$s_z$本征态$\chi_{1/2}(s_z)=\begin{pmatrix} 1 \\ 0 \end{pmatrix}$下,求 $\overline{(\delta{s_x})^2}$ 和 $\overline{(\delta{s_y})^2}$.
\end{question}
\begin{solution}

\end{solution}



\begin{question}{Ex103}
    设两电子在弹性辏力场中运动,每个电子的势能为
    $$
        u(r)=\frac{1}{2}\mu\omega^2r^2
    $$
    如果电子之间的库仑能与$u(r)$相比可以忽略,求当一个电子处在基态,另一电子处于沿$x$方向运动时的第一激发态时,两电子组成体系的波函数。
\end{question}
\begin{solution}
    设两个电子的坐标为$r_1$和$r_2$,则体系的总哈密顿算符为
    $$
        H=\frac{p_1^2}{2m}+\frac{p_2^2}{2m}+\frac{1}{2}m\omega^2(r_1^2+r_2^2)
    $$
    其中$m$是电子的质量,$\omega$是弹性势场的频率。可以用质心坐标$R=\frac{r_1+r_2}{2}$和相对坐标$r=r_1-r_2$将哈密顿算符分解为
    $$
        H=H_R+H_r=\frac{P_R^2}{4m}+\frac{p_r^2}{m}+m\omega^2R^2+\frac{1}{4}m\omega^2r^2
    $$
    其中$P_R=p_1+p_2$是质心动量,$p_r=\frac{p_1-p_2}{2}$是相对动量。由于质心运动和相对运动是分离的,可以分别求解它们的本征态和本征值。质心运动部分是一个一维谐振子,其本征态为
    $$
        \psi_n(R)=\left(\frac{m\omega}{\pi\hbar}\right)^{1/4}\frac{1}{\sqrt{2^n n!}}H_n(\sqrt{m\omega/\hbar}R)e^{-m\omega R^2/2\hbar}
    $$
    其中$n=0,1,2,\dots$,$H_n(x)$是$n$阶厄米多项式,本征值为
    $$
        E_n=\hbar\omega(n+1/2)
    $$
    相对运动部分是一个二维谐振子,其本征态为
    $$\
        phi_{nm}(r)=\left(\frac{m\omega}{\pi\hbar}\right)^{1/2}\frac{1}{\sqrt{2^n n! 2^m m!}}H_n(\sqrt{m\omega/\hbar}x)H_m(\sqrt{m\omega/\hbar}y)e^{-m\omega r^2/4\hbar}
    $$
    其中$n,m=0,1,2,\dots$,本征值为
    $$
        E_{nm}=2\hbar\omega(n+m+1)
    $$
    由于电子是费米子,它们的总波函数必须是反对称的。因此,如果一个电子处在基态$\psi_0(R)\phi_{00}(r)$,另一个电子处于沿$x$方向运动的第一激发态$\psi_0(R)\phi_{10}(r)$,则两电子组成体系的波函数为
    $$
        \Psi(r_1,r_2)=A[\psi_0(R)\phi_{00}(r)\chi_{+-}(s)-\psi_0(R)\phi_{10}(r)\chi_{-+}(s)]
    $$
    其中$A$是归一化常数,$\chi_{+-}(s)$和$\chi_{-+}(s)$是自旋单态波函数。这个波函数满足交换反对称性:
    $$
        \Psi(r_1,r_2)=-\Psi(r_2,r_1)
    $$
\end{solution}





