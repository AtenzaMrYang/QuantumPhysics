\section{量子力学的矩阵形式与表象变换}
希尔伯特空间中,态矢的内积运算(行矩阵乘列矩阵)定义为
$$
    \langle f \mid g \rangle=f_1g_2+f_2g_2+\cdots
$$
波函数的内积
$$
    \langle f \mid g \rangle=\int_{-\infty}^{+\infty}f^*(x)g(x)\,\mathrm{d}x
$$
且一个波函数的模平方积分就是与自身的内积,必须是一个有限值(否则无法归一化)
$$
    \int_{a}^{b}|f(x)|^2\,\mathrm{d}x < \infty
$$
态矢的正交归一表示
$$
    \langle f_m \mid f_n \rangle=\delta_{mn}
$$
广义的态叠加原理:对任意一个希尔伯特空间中的态$\mid\psi\rangle$而言,都可以写成某个力学量算符$\hat{Q}$的本征函数族的线性组合
$$
    \mid\psi\rangle = \sum_{n}^{}c_n\mid\psi_n\rangle
$$
举例来说,动量空间波函数和位置空间波函数都能描述同一个量子态,二者可以用傅里叶变换相互转化
$$
    \varphi(x, t)=\frac{1}{\sqrt{2\pi\hbar}}\int_{-\infty}^{+\infty}\psi(p, t)\mathrm{e}^{-\mathrm{i}px/\hbar}\,\mathrm{d}p
$$
$$
    \psi(p, t)=\frac{1}{\sqrt{2\pi\hbar}}\int_{-\infty}^{+\infty}\varphi(x, t)\mathrm{e}^{\mathrm{i}px/\hbar}\,\mathrm{d}x
$$
\begin{question}{题目7.3}
    设一维粒子Hamilton量为$H=\dfrac{p^2}{2m}+V(x)$. 写出 $x$ 表象中$x,p,H$的矩阵元.
\end{question}
\begin{solution}
    坐标$x$在坐标表象中表示为
    $$
        \langle x'|x|x'' \rangle = x'\delta(x'-x'')
    $$
    势能$V(x)$在坐标表象中表示为
    $$
        \langle x'|V(x)|x'' \rangle = V(x')\delta(x'-x'')
    $$
    由此可得各种矩阵元
    $$
        (x)_{x'x''} = \langle x'|x|x'' \rangle
        = \int\delta(x-x')x\delta(x-x'')\,\mathrm{d}x
        = x'\delta(x'-x'')
    $$
    $$
        \begin{aligned}
            (p)_{x'x''}
             & = \langle x'|p|x'' \rangle = \int\delta(x-x')\left(-\mathrm{i}\hbar\frac{\partial }{\partial x}\right)\delta(x-x'')\,\mathrm{d}x \\
             & = -\mathrm{i}\hbar\frac{\partial }{\partial x'}\int\delta(x-x')\delta(x-x'')\,\mathrm{d}x                                        \\
             & = -\mathrm{i}\hbar\frac{\partial }{\partial x'}\delta(x'-x'')
        \end{aligned}
    $$
    $$
        \begin{aligned}
            (H)_{x'x''}
             & =\langle x'|H|x'' \rangle =\int\delta(x-x')H\delta(x-x'')\,\mathrm{d}x                                          \\
             & =\int\delta(x-x')\left[-\frac{\hbar^2}{2m}\frac{\partial^2}{\partial x^2}+V(x)\right]\delta(x-x'')\,\mathrm{d}x \\
             & =-\frac{\hbar^2}{2m}\frac{\partial^2}{\partial x'^2}\delta(x'-x'') + V(x')\delta(x'-x'')                        \\
        \end{aligned}
    $$
\end{solution}


\begin{question}{习题集Ex94}
    已知在$\sigma^2-\sigma_z$表象中,算符$\hat{\sigma}_x$ 和 $\hat{\sigma}_y$的矩阵形式为
    $$
        \sigma_x=\begin{pmatrix}
            0 & 1 \\
            1 & 0
        \end{pmatrix}
        \quad
        \sigma_y=\begin{pmatrix}
            0          & -\mathrm{i} \\
            \mathrm{i} & 0
        \end{pmatrix}
    $$
    \begin{enumerate}
        \item 求它们的本征值与本征函数.
        \item 写出在$\sigma^2-\sigma_x$表象中,算符$\hat{\sigma}_x$的矩阵形式及其本征函数形式.
    \end{enumerate}
\end{question}
\begin{solution}
    (1) 设算符$\hat{\sigma}_x$的本征值为$\lambda$,由久期方程
    $$
        \begin{vmatrix}
            -\lambda & 1        \\
            1        & -\lambda
        \end{vmatrix}=0
        \implies
        \lambda_1=1, \lambda_2=-1
    $$
    \paragraph{情形一} 当$\lambda_1=1$时,设相应的$\hat{\sigma}_z$的本征函数为$\psi_1=(a_1, a_2)^T$
    $$
        \begin{pmatrix}
            0 & 1 \\
            1 & 0
        \end{pmatrix}\begin{pmatrix}
            a_1 \\
            a_2
        \end{pmatrix} = \begin{pmatrix}
            a_1 \\
            a_2
        \end{pmatrix}
    $$
    说明$a_1=a_2$,且波函数必须正交归一
    $$
        \psi^*\psi
        =\begin{pmatrix}
            a_1^* & a_2^*
        \end{pmatrix}\begin{pmatrix}
            a_1 \\
            a_2
        \end{pmatrix}
        =1 \implies a_1=a_2=\frac{1}{\sqrt{2}}
    $$
    \paragraph{情形二} 当$\lambda=-1$时,设算符$\hat{\sigma}_x$的本征函数为$\psi_2=(b_1, b_2)^T$,于是本征方程写为
    $$
        \begin{pmatrix}
            0 & 1 \\
            1 & 0 \\
        \end{pmatrix}\begin{pmatrix}
            b_1 \\
            b_2
        \end{pmatrix}
        =-\begin{pmatrix}
            b_1 \\
            b_2
        \end{pmatrix}
    $$
    这说明$b_1=-b_2$,且波函数必须正交归一
    $$
        \psi^*\psi
        =\begin{pmatrix}
            b_1^* & b_2^*
        \end{pmatrix}\begin{pmatrix}
            b_1 \\
            b_2
        \end{pmatrix}
        =1 \implies b_1=-b_2=\frac{1}{\sqrt{2}}
    $$
    综上:
    $$
        \lambda_1=1,\psi_1=\frac{1}{\sqrt{2}}\begin{pmatrix}1\\1\end{pmatrix};
        \quad
        \lambda_2=-1,\psi_2=\frac{1}{\sqrt{2}}\begin{pmatrix}1\\-1\end{pmatrix}
    $$
    同理,通过久期方程和本征方程,求得$\hat{\sigma}_y$的本征值和相应的本征函数
    $$
        \lambda_1'=1,\varphi_1=\frac{1}{\sqrt{2}}\begin{pmatrix}1\\\mathrm{i}\end{pmatrix};
        \quad
        \lambda_2'=-1,\varphi_2=\frac{1}{\sqrt{2}}\begin{pmatrix}1\\-\mathrm{i}\end{pmatrix}
    $$
    (2) 可以利用转移矩阵$U$将态矢从$\sigma^2-\sigma_z$表象变换到$\sigma^2-\sigma_x$表象
\end{solution}



\begin{question}{题目8.1}
    \begin{enumerate}
        \item 在$\sigma_z$表象中,求$\sigma_x$的本征态;
        \item 求$\sigma_z$表象变换到$\sigma_x$表象的变换矩阵;
        \item 验证
              $$
                  S\sigma_xS^{-1} = S\begin{pmatrix} 0 & 1 \\ 1 & 0 \end{pmatrix}S^{-1}=\begin{pmatrix} 1 & 0 \\ 0 & -1 \end{pmatrix}
              $$
    \end{enumerate}
\end{question}
\begin{solution}
\end{solution}



\begin{question}{题目8.3}
    在$s_z$本征态$\chi_{1/2}(s_z)=\begin{pmatrix} 1 \\ 0 \end{pmatrix}$下,求 $\overline{(\Delta{s_x})^2}$ 和 $\overline{(\Delta{s_y})^2}$.
\end{question}
\begin{solution}

\end{solution}



\begin{question}{Ex103}
    设两电子在弹性辏力场中运动,每个电子的势能为
    $$
        u(r)=\frac{1}{2}\mu\omega^2r^2
    $$
    如果电子之间的库仑能与$u(r)$相比可以忽略,求当一个电子处在基态,另一电子处于沿$x$方向运动时的第一激发态时,两电子组成体系的波函数。
\end{question}
\begin{solution}
\end{solution}





