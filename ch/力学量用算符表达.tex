\section{力学量用算符表达}

\subsection{算符的运算规则}
以下所有的讨论都要建立在\textbf{「对任意波函数和任意常数都成立」}这一前提上.

\subsubsection{线性算符}
\begin{equation}
    \hat{A}(c_1\psi_1+c_2\psi_2)=c_1\hat{A}\psi_1+c_2\hat{A}\psi_2
\end{equation}

\subsubsection{算符之和}
\begin{equation}
    \left(\hat{A}+\hat{B}\right)\psi = \hat{A}\psi+\hat{B}\psi
\end{equation}

\subsubsection{算符之积}
\begin{equation}
    \left(\hat{A}\hat{B}\right)\psi = \hat{A}\left(\hat{B}\psi\right)
\end{equation}

\subsubsection{量子力学的基本对易式}
量子力学中最基本的对易关系是\footnote{
    动量算符表示为
    $$
        \hat{p} = \mathrm{i}\hbar\left( \frac{\partial }{\partial x}\hat{e}_x + \frac{\partial }{\partial y}\hat{e}_y + \frac{\partial }{\partial z}\hat{e}_y \right)
    $$
    Kronecker函数的含义是
    $$
        \delta_{ij}=\begin{cases}
            0, & i \neq j \\
            1, & i = j    \\
        \end{cases}
    $$
}
$$
    x_{\alpha}\hat{p}_{\beta}-\hat{p}_{\beta}x_{\alpha}=\mathrm{i}\hbar\delta_{\alpha\beta}, \quad \alpha,\beta=x,y,z
$$
抽象但普遍的量子力学对易式
\begin{equation}
    \left[\hat{A}, \hat{B}\right] \equiv \hat{A}\hat{B}-\hat{B}\hat{A}
\end{equation}
而对易式满足下列恒等式:
\begin{align}
    \left[\hat{A}, \hat{B}\right]         & = -\left[\hat{B}, \hat{A}\right]                                              \\
    \left[\hat{A}, \hat{B}+\hat{C}\right] & = \left[\hat{A}, \hat{B}\right] + \left[\hat{A}, \hat{C}\right]               \\
    \left[\hat{A}, \hat{B}\hat{C}\right]  & = \hat{B}\left[\hat{A}, \hat{C}\right] + \left[\hat{A}, \hat{B}\right]\hat{C} \\
    \left[\hat{A}\hat{B}, \hat{C}\right]  & = \hat{A}\left[\hat{B}, \hat{C}\right] + \left[\hat{A}, \hat{C}\right]\hat{B}
\end{align}
进一步可以证明Jacobi恒等式\footnote{李代数是满足Jacobi恒等式代数结构的一个主要例子}
\begin{equation}
    \left[\hat{A}, \left[\hat{B},\hat{C}\right]\right] + \left[\hat{B}, \left[\hat{C},\hat{A}\right]\right] + \left[\hat{C}, \left[\hat{A},\hat{B}\right]\right] = 0
\end{equation}



\subsubsection{角动量的对易式}
角动量算符定义为
$$
    \hat{\boldsymbol{l}} = \boldsymbol{r} \times \hat{\boldsymbol{p}}
$$
各个分量表示为
$$
    \begin{aligned}
        \hat{l}_{x} = y\hat{p}_z-z\hat{p}_y = -\mathrm{i}\hbar\left(y\frac{\partial }{\partial z} - z\frac{\partial }{\partial y}\right) \\
        \hat{l}_{y} = z\hat{p}_x-x\hat{p}_z = -\mathrm{i}\hbar\left(z\frac{\partial }{\partial x} - x\frac{\partial }{\partial z}\right) \\
        \hat{l}_{z} = x\hat{p}_y-y\hat{p}_x = -\mathrm{i}\hbar\left(x\frac{\partial }{\partial y} - y\frac{\partial }{\partial x}\right) \\
    \end{aligned}
$$
不难证明
\begin{equation}
    \left[\hat{l}_{\alpha}, x_{\beta}\right] = \varepsilon_{\alpha\beta\gamma}\mathrm{i}\hbar x_{\gamma}
\end{equation}
上式中的$\varepsilon_{\alpha\beta\gamma}$称为Levi-Civita符号
\begin{equation}\label{levi-civita}
    \begin{cases}
        \varepsilon_{\alpha\beta\gamma} = -\varepsilon_{\beta\alpha\gamma} = -\varepsilon_{\alpha\gamma\beta} \\
        \varepsilon_{123} = 1                                                                                 \\
    \end{cases}
\end{equation}
类似地,还可以证明
\begin{align}
    \left[\hat{l}_{\alpha}, \hat{p}_{\beta}\right] & = \varepsilon_{\alpha\beta\gamma}\mathrm{i}\hbar\hat{p}_{\gamma} \\
    \left[\hat{l}_{\alpha}, \hat{l}_{\beta}\right] & = \varepsilon_{\alpha\beta\gamma}\mathrm{i}\hbar\hat{l}_{\gamma}
\end{align}


\begin{question}{证明题}定义$\hat{\boldsymbol{l}}^2 = \hat{l}_x^2 + \hat{l}_y^2 + \hat{l}_z^2$,试证明:
    $$
        \left[\hat{\boldsymbol{l}}^2, \hat{l}_{\alpha}\right] = 0,
        \quad \alpha=x,y,z
    $$
\end{question}
\begin{solution}
    先证明$\alpha=x$的情形
    $$
        \begin{aligned}
            LHS
             & = \left[\hat{\boldsymbol{l}}^2, \hat{l}_x\right] = \left[\left(\hat{l}_x^2+\hat{l}_y^2+\hat{l}_z^2\right), \hat{l}_x\right] \\
             & = \left[\hat{l}_x^2, \hat{l}_x\right] + \left[\hat{l}_y^2, \hat{l}_x\right] + \left[\hat{l}_z^2, \hat{l}_x\right]           \\
             & = \hat{l}_x^3-\hat{l}_x^3+\hat{l}_y^2\hat{l}_x-\hat{l}_x\hat{l}_y^2+\hat{l}_z^2\hat{l}_x-\hat{l}_x\hat{l}_z^2               \\
             & = 0
        \end{aligned}
    $$
    可见$LHS=RHS$,同理也可以证明
    $$
        \left[\hat{\boldsymbol{l}}^2, \hat{l}_y\right] = 0,
        \quad
        \left[\hat{\boldsymbol{l}}^2, \hat{l}_z\right] = 0.
    $$
    这说明角动量的平方$\hat{\boldsymbol{l}}^2$与角动量的任意分量$\hat{l}_{\alpha}$($\alpha=x, y, z$)相互对易.
\end{solution}




\begin{question}{证明题}
    定义
    $$
        \hat{l}_{\pm} = \hat{l}_x \pm \mathrm{i}\hat{l}_y
    $$
    证明:
    \begin{enumerate}
        \item[(19)] $\hat{l}_z\hat{l}_{\pm}=\hat{l}_{\pm}\left(\hat{l}_z \pm \hbar\right)$
        \item[(20)] $\hat{l}_{\pm}\hat{l}_{\mp}=\hat{l}^2 - \hat{l}_z^2 \pm \hbar\hat{l}_z$
        \item[(21)] $\left[\hat{l}_{+}, \hat{l}_{-}\right]=2\hbar\hat{l}_z$
    \end{enumerate}
\end{question}
\begin{solution}
    (19) 根据$\hat{l}_{\pm}$的定义
    $$
        LHS = \hat{l}_z\hat{l}_{\pm}
        = \hat{l}_z\left(\hat{l}_x \pm \mathrm{i}\hat{l}_y\right)
        = \hat{l}_z\hat{l}_x \pm \mathrm{i}\hat{l}_z\hat{l}_y
    $$
    $$
        RHS = \left(\hat{l}_x \pm \mathrm{i}\hat{l}_y\right)\left(\hat{l}_z \pm \hbar\right)
    $$
    所以
    $$
        RHS=LHS
    $$
    (20) 根据$\hat{l}_{\pm}$的定义
    $$
        LHS =
    $$
    $$
        RHS =
    $$
    所以
    $$
        RHS=LHS
    $$
    (21) 根据$\hat{l}_{\pm}$的定义
    $$
        \begin{aligned}
            LHS & = [\hat{l}_x+\mathrm{i}\hat{l}_y,\hat{l}_x-\mathrm{i}\hat{l}_y]                                                                       \\
                & = [\hat{l}_x,\hat{l}_x-\mathrm{i}\hat{l}_y]+[\mathrm{i}\hat{l}_y,\hat{l}_x-\mathrm{i}\hat{l}_y]                                       \\
                & = [\hat{l}_x,\hat{l}_x]-[\hat{l}_x,\mathrm{i}\hat{l}_y]+[\mathrm{i}\hat{l}_y,\hat{l}_x]-[\mathrm{i}\hat{l}_y,\mathrm{i}\hat{l}_y]     \\
                & = 0-\mathrm{i}[\hat{l}_x,\hat{l}_y]-[\hat{l}_x,\mathrm{i}]\hat{l}_y+\mathrm{i}[\hat{l}_y,\hat{l}_x]+[\hat{l}_y,\mathrm{i}]\hat{l}_x-0 \\
                & = 0-\mathrm{i}(\mathrm{i}\hbar\hat{l}_z)-0+\mathrm{i}(-\mathrm{i}\hbar\hat{l}_z)+0-0                                                  \\
                & = 2\hbar\hat{l}_z = RHS                                                                                                               \\
        \end{aligned}
    $$
\end{solution}

\subsubsection{逆算符}
$$
    \hat{A}\psi=\phi \iff \hat{A}^{-1}\phi=\psi
$$

\subsubsection{算符的函数}
类比泰勒级数展开,可以得到算符$\hat{A}$的函数
\begin{equation}
    F\left(\hat{A}\right)=\sum_{n=0}^{\infty}\frac{F^{(n)}(0)}{n!}\hat{A}^n
\end{equation}
进一步,还能得到位移算符
$$
    \mathrm{e}^{a\frac{\mathrm{d}}{\mathrm{d}x}}\psi(x) = \psi(x+a)
$$

\subsubsection{转置算符}
算符$\hat{A}$的转置算符定义为
$$
    \int\mathrm{d}\tau \,\psi^*\tilde{\hat{A}}\varphi = \int\mathrm{d}\tau \,\varphi\hat{A}\psi
$$
表示为
\begin{equation}
    \left(\psi, \tilde{\hat{A}}\varphi\right) = \left(\varphi^*, \hat{A}\psi^*\right)
\end{equation}



\subsubsection{复共轭算符与Hermitian共轭算符}
算符$\hat{A}$的复共轭算符定义为
\begin{equation}
    \hat{A}^*\psi = \left(\hat{A}\psi^*\right)^*
\end{equation}
算符$\hat{A}$的Hermitian共轭算符定义为
\begin{equation}
    \left(\psi, \hat{A}^{\dagger}\varphi\right) = \left(\hat{A}\psi, \varphi\right)
\end{equation}

\subsubsection{转置并取复共轭}
$$
    \langle\varphi|\hat{A}|\psi\rangle=\overline{\langle\psi|\hat{B}|\varphi\rangle}
$$

\begin{question}{题目3.1}
    设$A$与$B$为Hermitian算符,则$\dfrac{1}{2}(AB+BA)$和$\dfrac{1}{2\mathrm{i}}(AB-BA)$也是Hermitian算符. 由此证明:任何一个算符$F$均可分解为$F=F_{+}+\mathrm{i}F_{-}$
    $$
        F_{+}=\frac{1}{2}(F+F^{\dagger}),\quad
        F_{-}=\frac{1}{2\mathrm{i}}(F-F^{\dagger})
    $$
    $F_{+}$与$F_{-}$均为Hermitian算符.
\end{question}
\begin{solution}
    Hermitian算符就是一种自伴算符
    $$
        A^\dagger = A, \quad B^\dagger = B, \quad (AB)^{\dagger}=B^{\dagger}A^{\dagger}=BA
    $$
    对于
    $$
        \frac{1}{2}(AB+BA)^\dagger
        = \frac{1}{2}\left(B^\dagger A^\dagger + A^\dagger B^\dagger\right)
        = \frac{1}{2}(BA+AB)
        = \frac{1}{2}(AB+BA)
    $$
    $$
        \left[\frac{1}{2\mathrm{i}}(AB-BA)\right]^\dagger
        = -\frac{1}{2\mathrm{i}}\left(B^{\dagger}A^{\dagger} - A^{\dagger}B^{\dagger}\right)
        = -\frac{1}{2\mathrm{i}}\left(BA-AB\right)
        = \frac{1}{2\mathrm{i}}\left(AB-BA\right)
    $$
    这说明以上两个算符都是Hermitian算符,根据这一结论,我们再对任一算符$F$分解
    $$
        F = \frac{F}{2}+\frac{F}{2}
        = \frac{1}{2}\left(F+F^\dagger\right)+\frac{1}{2}\left(F-F^\dagger\right)
        = F_{\dagger}+\mathrm{i}F_{-}
    $$
    这说明$F_{+}$和$F_{-}$都是Hermitian算符.
\end{solution}



\begin{question}{题目3.4}
    定义反对易式
    $$
        [A, B]_{\dagger} \equiv AB+BA
    $$
    再证明
    $$
        \begin{aligned}
             & [AB, C] = A[B, C]_{\dagger} - [A, C]_{\dagger}B \\
             & [A, BC] = [A, B]_{\dagger}C - B[A, C]_{\dagger} \\
        \end{aligned}
    $$
\end{question}
\begin{solution}
    (1) 根据反对易式的定义
    $$
        \begin{cases}
            LHS=A[B,C]+[A,C]B=A(BC-CB)+(AC-CA)B=ABC-CAB \\
            RHS=A(BC+CB)-(AC+CA)B=ABC-CAB
        \end{cases}
    $$
    (2) 类似地
    $$
        \begin{cases}
            LHS=B[A,C]+[A,B]C=B(AC-CA)+(AB-BA)C=ABC-BCA \\
            RHS=(AB+BA)C-B(AC+CA)=ABC-BCA
        \end{cases}
    $$
    以上两式均有$LHS=RHS$,证毕.
\end{solution}

\subsection{Hermitian算符的本征值与本征函数}

\subsubsection{涨落}
对于处于量子态$\psi$的体系,力学量$A$的测量结果会围绕平均值涨落,即:
$$
    \overline{\Delta A^2} = \overline{\left(\hat{A}-\overline{A}\right)^2}
    = \int_{}^{}\psi^*\left(\hat{A}-\overline{A}\right)^2\psi\,\mathrm{d}\tau
$$
因为$\hat{A}$是Hermitian算符,$\overline{A}$必为实数,因此$\overline{\Delta A}=\left(\hat{A}-\overline{A}\right)$仍然是Hermitian算符
$$
    \overline{\Delta A^2} = \int\left|\left(\hat{A}-\overline{A}\right)\psi\right|^2\,\mathrm{d}\tau \geqslant 0
$$

\subsubsection{本征态}
我们把测量结果永远不变的特殊量子态被称为本征态
$$
    \overline{\Delta A^2}=0 \implies \left(\hat{A}-\overline{A}\right)\psi = 0
$$
记常数$\overline{A}$为$A_n$,本征态为$\psi_n$,整理得到算符$\hat{A}$的本征方程
$$
    \hat{A}\psi_n = A_n\psi_n
$$
其中$A_n$称为算符$\hat{A}$的一个本征值,$\psi_n$为相应的本征态.




\begin{theorem}
    Hermitian算符的本征值必为实.
\end{theorem}
\begin{proof}
    在本征态$\psi_n$下
    $$
        \overline{A} = \left(\psi_n, \hat{A}\psi_n\right) = A_n(\psi_n,\psi_n) = A_n
    $$
    因为$\overline{A}$必为实数,所以本征值$A_n$也一定是实数.
\end{proof}




\begin{theorem}
    Hermitian算符的属于不同本征值的本征函数,彼此正交.
\end{theorem}
\begin{proof}
    设量子态$\psi$的两个本征值分别为$A_m$和$A_n$,对应的本征函数分别为$\psi_m$和$\psi_n$
\end{proof}


\subsubsection{本征态简并}
本征态简并往往与体系的对称性密切相关,在能级简并的情况下,仅根据能量本征值并不能完全确定各能量简并态.

设力学量$\hat{A}$的本征方程为
$$
    \hat{A}\psi_{n\alpha}=A_n\psi_{n\alpha}, \quad \alpha =1, 2, \cdots, f_n
$$
即属于本征值$A_n$的本征态有$f_n$个,本征值$A_n$为$f_n$重简并.

在出现简并态时,简并态的选择并不唯一,而且往往也不彼此正交,但只要把它们适当地线性叠加,就能获得一组彼此正交的简并态
$$
    \phi_{n\beta} = \sum_{a}^{f_n} \alpha_{\beta\alpha}\hat{A}\psi_{n\alpha}, \quad \beta = 1, 2, \cdots, f_{\alpha}
$$
容易证明$\phi_{n\beta}$仍为$\hat{A}$的本征态,相应的本征值仍为$A_n$,因为
$$
    \hat{A}\phi_{n\beta}
    = \sum_{a}a_{\beta\alpha}\hat{A}\psi_{n\alpha}
    = A_n\sum_{a}a_{\beta\alpha}\psi_{n\alpha}
    = A_n\phi_{n\beta}
$$
问题转化为:能否找到合适的$\alpha_{\beta\alpha}$,使$\phi_{n\beta}$具有正交性?也即下列方程组是否有解的问题
$$
    \left(\phi_{n\beta}, \phi_{n\beta'}\right)=\delta_{\beta\beta'}
$$
实际上,我们目前有$\dfrac{1}{2}f_n(f_n-1)+f_n=\dfrac{1}{2}f_n(f_n+1)$个线性方程,而待解系数$a_{\beta\alpha}$只有$f_n^2$个,根据
$$
    f_n^2 \geqslant \frac{1}{2}f_n(f_n+1)
$$
可以说明线性方程组个数多于系数个数,方程组显然有解,也即一定能找到合适的$a_{\beta\alpha}$使正交性条件得到满足.

\begin{question}{习题集Ex.43}
    求算符$\displaystyle \hat{F}=-\mathrm{ie}^{\mathrm{i}x}\frac{\mathrm{d}}{\mathrm{d}x}$的本征函数
\end{question}
\begin{solution}
    设量子本征态为$\psi_n(x)$,本征值为$F_n$,本征方程为
    $$
        \hat{F}\psi_n(x) = -\mathrm{ie}^{\mathrm{i}x}\frac{\mathrm{d}\psi_n(x)}{\mathrm{d}x} = F_n\psi_n(x)
    $$
    分离变量并两边积分,得到
    $$
        \psi_n(x)=C\exp\left(-F_n\mathrm{e}^{-\mathrm{i}x}\right)
    $$
\end{solution}



\begin{question}{习题集Ex.44}
    对于一维运动,求算符$\hat{F}=\hat{p}+x$的本征值和本征函数.
\end{question}
\begin{solution}
    设量子本征态为$\psi_n(x)$,本征值为$F_n$,本征方程为
    $$
        \hat{F}\psi_n(x) = (\hat{p}_x+x)\psi_n(x) = F_n\psi_n(x)
    $$
    代入动量算符,得到
    $$
        -\mathrm{i}\hbar\frac{\mathrm{d}\psi_n(x)}{\mathrm{d}x}+(x-F_n)\psi_n(x)=0
    $$
    分离变量后两边积分得到本征函数
    $$
        \psi_n(x) = C\exp\left[-\frac{\mathrm{i}x}{2\hbar}(x-2F_n)\right]
    $$
\end{solution}



\begin{question}{习题集Ex.48}
    若算符$\hat{K}$有属于本征值为$\lambda$的本征函数$\phi$,且有$\hat{K}=\hat{A}\hat{B}$和$\hat{A}\hat{B}-\hat{B}\hat{A}=1$,试证明$u_1=\hat{A}\phi$和$u_2=\hat{B}\phi$也是算符$\hat{K}$的本征函数,且对应的本征值分别为$\lambda-1$和$\lambda+1$.
\end{question}
\begin{solution}
    (1) 对于算符$\hat{K}$和函数$u_1=\hat{A}\phi$
    $$
        \hat{K}u_1=\hat{K}\hat{A}\phi=\hat{A}\hat{B}\hat{A}\phi=\hat{A}\left(\hat{A}\hat{B}-1\right)\phi=\hat{A}\left(\hat{K}-1\right)\phi
    $$
    设本征值为$E_1$,则本征方程为
    $$
        \hat{A}\left(\hat{K}-1\right)\phi = \hat{A}\left(\lambda-1\right)\phi = E_1\hat{A}\phi
    $$
    两边左乘$\hat{A}^{-1}$,于是
    $$
        E_1=\lambda-1
    $$
    (2) 对于算符$\hat{K}$和函数$u_2=\hat{B}\phi$
    $$
        \hat{K}u_2=\hat{A}\hat{B}\hat{B}\phi=\left(1+\hat{B}\hat{A}\right)\hat{B}\phi=\hat{B}\left(1+\hat{A}\hat{B}\right)\phi=\hat{B}\left(1+\hat{K}\right)\phi
    $$
    设本征值为$E_2$,则本征方程为
    $$
        \hat{B}\left(1+\hat{K}\right)\phi = \hat{B}\left(1+\lambda\right)\phi = E_2\hat{B}\phi
    $$
    两边左乘$\hat{B}^{-1}$,于是
    $$
        E_2=\lambda+1
    $$
\end{solution}





\subsection{共同本征函数}

\begin{question}{题目3.14}
    证明在$l_z$的本征态下,$\overline{l_x}=\overline{l_y}=0$. (提示:利用$l_yl_z-l_zl_y=\mathrm{i}\hbar l_x$,求平均)
\end{question}
\begin{solution}
    对于球谐函数$Y_{lm}$
    $$
        l_zY_{lm} = m\hbar Y_{lm}, \quad m = l, l-1, \cdots, -l+1, -l
    $$
    设$\psi_m$是$l_z$的本征态,且相应的本征值是$m\hbar$
    $$
        l_z\psi_m = m\hbar\psi_m
    $$
    根据角动量的对易关系
    $$
        l_yl_z-l_zl_y=\mathrm{i}\hbar l_x
    $$
    得到
    $$
        \begin{aligned}
            \overline{l_x}
             & = \frac{1}{\mathrm{i}\hbar}\int\psi_m^*\left(l_yl_z - l_zl_y\right)\psi_m \,\mathrm{d}x                                          \\
             & = \frac{1}{\mathrm{i}\hbar}\left[\int\psi_m^*l_yl_z\psi_m\,\mathrm{d}x - \int\psi_m^*l_zl_y\psi_m \,\mathrm{d}x\right]           \\
             & = \frac{1}{\mathrm{i}\hbar}\left[m\hbar\int\psi_m^*l_y\,\mathrm{d}x - \int\left(l_z\psi_m\right)^*l_y\psi_m \,\mathrm{d}x\right] \\
             & = \frac{1}{\mathrm{i}\hbar}m\hbar\left[\overline{l_y} - \overline{l_z}\right]                                                    \\
             & = 0                                                                                                                              \\
        \end{aligned}
    $$
\end{solution}




\begin{question}{题目3.15}
    设粒子处于$Y_{lm}(\theta,\varphi)$状态下,求$\overline{(\Delta l_x)^2}$和$\overline{(\Delta l_y)^2}$.
\end{question}
\begin{solution}
    球谐函数$Y_{lm}(\theta, \varphi)$是$\boldsymbol{l}^2$和$l_z$的本征函数
    $$
        \begin{aligned}
            \boldsymbol{l}^2Y_{lm}(\theta, \varphi) & = l(l+1)\hbar^2Y_{lm}(\theta, \varphi), \\
            l_zY_{lm}(\theta, \varphi)              & = m\hbar^2Y_{lm}(\theta, \varphi)
        \end{aligned}
    $$
    考虑到球谐函数的对称性,有
    $$
        \overline{l_x} = \overline{l_y} = 0
    $$
    $$
        \overline{l_x^2} = \overline{l_y^2} = \frac{1}{2}\overline{\left(\boldsymbol{l}^2-l_z^2\right)} = \frac{\hbar^2}{2}\left[l(l+1)-m^2\right]
    $$
    根据概率统计的相关知识
    $$
        \overline{(\Delta l_x)^2} = \overline{(\Delta l_z)^2} = \frac{\hbar^2}{2}\left[l(l+1)-m^2\right]
    $$
\end{solution}





\begin{question}{题目3.16}
    设体系处于$\psi=c_1Y_{11}+c_2Y_{20}$状态(已归一化,即$|c_1|^2+|c_2|^2=1$),求:
    \begin{enumerate}
        \item $l_z$的可能测值及平均值;
        \item $\boldsymbol{l}^2$的可能测值及相应的概率;
        \item $l_x$的可能测值及相应的概率.
    \end{enumerate}
\end{question}
\begin{solution}
    按照态叠加原理,体系的任何一个状态$\psi$都可以用$\{\phi_{\alpha}\}$展开,并且利用$\psi_{\alpha}$的正交归一性,可以求出展开系数$a_{\alpha}$
    $$
        \psi = \sum_{\alpha} a_{\alpha}\psi_{\alpha}
    $$
    (1) $l_z$的可能测值和对应概率为
    $$
        \begin{aligned}
             & E_{11} = \hbar, & P_{11} = |c_1|^2 \\
             & E_{20} = 0,     & P_{20} = |c_2|^2
        \end{aligned}
    $$
    (2) $\boldsymbol{l}^2$的可能测值和对应概率为
    $$
        \begin{aligned}
             & E_{11} = 2\hbar^2, & P_{11} = |c_1|^2 \\
             & E_{20} = 6\hbar,   & P_{20} = |c_2|^2
        \end{aligned}
    $$
\end{solution}





\begin{question}{习题Ex55}
    线性谐振子在初始时刻处于下面归一化状态:
    $$
        \psi(x) = \sqrt{\frac{1}{5}}\psi_0(x) + \sqrt{\frac{1}{2}}\psi_2(x) + c_5\psi_5(x)
    $$
    式中$\psi_n(x)$表示谐振子第 $n$ 个定态波函数,求:
    \begin{enumerate}
        \item[(1)] 系数 $c_5$;
        \item[(2)] $t$时刻的波函数;
        \item[(3)] $t=0$时刻谐振子能量的可能取值及其相应几率,并求其平均值;
        \item[(4)] $t$时刻谐振子能量的可能取值及其相应的几率,并求其平均值.
    \end{enumerate}
\end{question}
\begin{solution}
    (1) 利用归一化条件
    $$
        \left(\sqrt{\frac{1}{5}}\right)^2 + \left(\sqrt{\frac{1}{2}}\right)^2 + c_5^2 = 1
    $$
    得到
    $$
        c_5 = \sqrt{\frac{3}{10}}
    $$
    (2) 定态波函数为
    $$
        \psi_n(x, t) = \psi_n(x)\mathrm{e}^{-\frac{\mathrm{i}}{\hbar}E_nt}
    $$
    而 $t$ 时刻的波函数为
    $$
        \psi_n(x, t) = \sqrt{\frac{1}{5}}\psi_0(x)\mathrm{e}^{-\frac{\mathrm{i}}{2}\omega t} + \sqrt{\frac{1}{2}}\psi_2(x)\mathrm{e}^{-\frac{5\mathrm{i}}{2}\omega t} + \sqrt{\frac{3}{10}}\psi_5(x)\mathrm{e}^{-\frac{11\mathrm{i}}{2}\omega t}
    $$
\end{solution}

