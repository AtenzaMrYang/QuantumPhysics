\section{力学量随时间的演化与对称性}
\subsection{守恒量}
力学量$A$的平均值表示为\footnote{波函数的标积定义为$$\left(\psi, \varphi\right) = \int_{}^{}\psi^*\varphi \,\mathrm{d}\tau$$}
$$
    \bar{A}(t) = \left(\psi(t), A\psi(t)\right)
$$
所以
$$
    \frac{\mathrm{d}}{\mathrm{d}t}\bar{A}(t)
    =\left(\frac{\partial \psi}{\partial t}, A\psi\right) + \left(\psi, A\frac{\partial \psi}{\partial t}\right) + \left(\psi, \frac{\partial A}{\partial t}\psi\right) \\
$$
利用薛定谔方程
$$
    \mathrm{i}\hbar\frac{\partial \psi}{\partial t} = \hat{H}\psi
$$
对$\displaystyle \frac{\partial \psi}{\partial t}$进行代换
$$
    \begin{aligned}
        \frac{\mathrm{d}}{\mathrm{d}t}\bar{A}(t)
         & =\left(\frac{H\psi}{\mathrm{i}\hbar}, A\psi\right)+\left(\psi, A\frac{H\psi}{\mathrm{i}\hbar}\right) + \left(\psi, \frac{\partial A}{\partial t}\psi\right) \\
         & =\frac{1}{-\mathrm{i}\hbar}(\psi, HA\psi) + \frac{1}{\mathrm{i}\hbar}(\psi, AH\psi) + \left(\psi, \frac{\partial A}{\partial t}\psi\right)                  \\
         & =\frac{1}{\mathrm{i}\hbar}(\psi, [A,H]\psi) +  \left(\psi, \frac{\partial A}{\partial t}\psi\right)                                                         \\
         & =\frac{1}{\mathrm{i}\hbar}\overline{[A,H]}+\frac{\overline{\partial A}}{\partial t}
    \end{aligned}
$$
如果力学量$A$不显含$t$,则$\displaystyle \frac{\partial A}{\partial t}=0$
\begin{equation}
    \frac{\mathrm{d}}{\mathrm{d}t}\bar{A}(t)=\frac{1}{\mathrm{i}\hbar}\overline{[A,H]}
\end{equation}
如果恰好有$[A, H]=0$(即力学量$A$恰好与$H$对易),则$\displaystyle \frac{\mathrm{d}}{\mathrm{d}t}\bar{A}=0$,这说明力学量$A$在任何量子态下的平均值都不会随时间改变,这是体系的一个守恒量.


\begin{question}{题目4.2}
    设体系有两个粒子,每个粒子可处于三个单粒子态$\varphi_1$,$\varphi_2$,$\varphi_3$中的任何一个态。试求体系可能的态数目,分三种情况讨论:两个全同玻色子、两个全同费米子、两个不同粒子。
\end{question}
\begin{solution}
    两个全同玻色子:对于全同玻色子,每个粒子都可以处于三个单粒子态中的任意一个,且两个粒子可以占据同一个态。因此,对于每个粒子,有三种选择,总共有 $3 \times 3 = 9$ 种可能的组合。然而,由于玻色子是全同的,这些组合中的重复状态需要被排除。我们知道,当两个玻色子占据同一个态时,它们是不可分辨的,因此只计算一次。因此,最终可能的态数目为 $9-1=8$。

    两个全同费米子:对于全同费米子,根据泡利不相容原理,两个费米子不能同时占据同一个态。因此,对于第一个粒子,有三种选择,而对于第二个粒子,只剩下两种选择(除去第一个粒子占据的态)。因此,可能的态数目为 $3 \times 2 = 6$。

    两个不同粒子:对于不同的粒子,每个粒子都可以处于三个单粒子态中的任意一个,且它们之间没有交换对称性的限制。因此,对于每个粒子,有三种选择,总共有 $3 \times 3 = 9$ 种可能的组合。由于粒子是不同的,这些组合中的重复状态不需要排除。因此,可能的态数目为 $9$。

    所以,根据上述讨论,根据粒子的全同性质,两个粒子的可能态数目分别为:
    \begin{itemize}
        \item 全同玻色子:8个态
        \item 全同费米子:6个态
        \item 不同粒子:9个态
    \end{itemize}
\end{solution}


\begin{question}{题目4.3}

\end{question}
\begin{solution}
    如果不考虑波函数的交换对称性,其可能的态数目为
    $$
        3^3=27
    $$
    如果要求波函数是交换反对称的,其可能的态数目为
    $$
        1
    $$
    如果要求波函数是交换对称的,其可能的态数目为
    $$
        1+6+3=10
    $$
\end{solution}






\begin{question}{题目4.4}
    设力学量 $A$ 不显含 $t$,$H$为体系的Hamilton量,证明:
    $$
        -\hbar^2\frac{\mathrm{d}^2}{\mathrm{d}t^2}\bar{A} = \overline{[[A, H], H]}.
    $$
\end{question}
\begin{solution}
    因为力学量$A$不显含$t$
    $$
        \frac{\mathrm{d}}{\mathrm{d}t}\bar{A}=\frac{1}{\mathrm{i}\hbar}\overline{[A, H]}
    $$
    上式两边再对$t$求导,则有
    $$
        \frac{\mathrm{d}^2}{\mathrm{d}t^2}\bar{A}
        =\frac{\mathrm{d}}{\mathrm{d}t}\frac{1}{\mathrm{i}\hbar}\overline{[A, H]}
        =\frac{1}{\mathrm{i}\hbar}\overline{\left[\frac{1}{\mathrm{i}\hbar}[A, H], H\right]}
        =-\frac{1}{\hbar^2}\overline{\left[[A, H], H\right]}
    $$
    简单整理得到
    $$
        -\hbar^2\frac{\mathrm{d}^2}{\mathrm{d}t^2}\bar{A} = \overline{[[A, H], H]}
    $$
    证毕.
\end{solution}




\begin{question}{题目4.5}
    设力学量$A$不显含$t$,证明在束缚定态下
    $$
        \frac{\mathrm{d}\bar{A}}{\mathrm{d}t} = 0
    $$
\end{question}
\begin{solution}
    定态$\psi$是体系的能量本征态,且束缚态可以归一化
    $$
        (\psi, \psi) = \text{有限值}
    $$
    因为力学量$A$不显含$t$
    $$
        \frac{\mathrm{d}\bar{A}}{\mathrm{d}t}=\frac{1}{\mathrm{i}\hbar}\overline{[A, H]}
    $$
    所以
    $$
        \frac{\mathrm{d}\bar{A}}{\mathrm{d}t}
        =\frac{1}{\mathrm{i}\hbar}\frac{(\psi, [A,H]\psi)}{(\psi, \psi)}
        =\frac{1}{\mathrm{i}\hbar}\frac{(\psi, AH\psi)-(\psi, HA\psi)}{(\psi, \psi)}
    $$
    利用能量本征方程和Hermite算符的性质
    $$
        \begin{aligned}
             & (\psi, AH\psi) = (\psi, AE\psi) \\
             & (\psi, HA\psi) = (H\psi, A\psi)
        \end{aligned}
    $$
    得到
    $$
        \frac{\mathrm{d}\bar{A}}{\mathrm{d}t}
        =\frac{1}{\mathrm{i}\hbar}\frac{(\psi, AE\psi)-(H\psi, A\psi)}{(\psi, \psi)}
        =\frac{E}{\mathrm{i}\hbar}\frac{(\bar{A}-\bar{A})}{(\psi, \psi)}
        =0
    $$
    证毕.
\end{solution}





