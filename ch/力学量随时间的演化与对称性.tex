\section{力学量随时间的演化与对称性}
\subsection{守恒量}
力学量$A$的平均值$\overline{A}$表示为\footnote{
    虽然在《量子力学教程》中,波函数的标积表示为
    $$
        (\psi, \varphi)=\int\psi^*\varphi\,\mathrm{d}\tau
    $$
    但最好还是用Dirac符号表示
    $$
        \langle\psi\mid\varphi\rangle=\int\psi^*\varphi\,\mathrm{d}\tau
    $$
}
$$
    \overline{A}=\langle\psi(t)\mid\hat{A}\psi(t)\rangle=\int\psi^*\hat{A}\psi\,\mathrm{d}\tau
$$
所以
$$
    \frac{\mathrm{d}}{\mathrm{d}t}\overline{A}
    =\frac{\mathrm{d}}{\mathrm{d}t}\int\psi^*\hat{A}\psi\,\mathrm{d}\tau
    =\int\frac{\mathrm{d}}{\mathrm{d}t}(\psi^*\hat{A}\psi)\,\mathrm{d}\tau
    =\int\left(\dot{\psi^*}\hat{A}\psi + \psi^*\dot{\hat{A}}\psi + \psi^*\hat{A}\dot{\psi}\right)\,\mathrm{d}\tau
$$
利用薛定谔方程做代换$\frac{\partial \psi}{\partial t}=\frac{\hat{H}}{\mathrm{i}\hbar}\psi$
$$
    \frac{\mathrm{d}}{\mathrm{d}t}\overline{A}
    =\int\left( \frac{\hat{H}\psi^*}{-\mathrm{i}\hbar}\hat{A}\psi
    + \psi^*\frac{\partial \hat{A}}{\partial t}\psi
    + \psi^*\hat{A}\frac{\hat{H}\psi}{\mathrm{i}\hbar}
    \right)\mathrm{d}\tau
$$
如果力学量$A$不显含$t$(也即$\frac{\partial A}{\partial t}=0$),上式还能消去一项
\begin{equation}
    \frac{\mathrm{d}}{\mathrm{d}t}\bar{A}
    =\frac{1}{\mathrm{i}\hbar}\int\left(-\hat{H}\psi^*\hat{A}\psi + \psi^*\hat{A}\hat{H}\psi\right)\,\mathrm{d}\tau
    =\frac{1}{\mathrm{i}\hbar}\int\psi^*(\hat{A}\hat{H}-\hat{H}\hat{A})\psi\,\mathrm{d}\tau
    =\frac{1}{\mathrm{i}\hbar}\overline{[A,H]}
\end{equation}
如果恰好有$[A, H]=0$(即力学量$A$恰好与$H$对易),则$\frac{\mathrm{d}}{\mathrm{d}t}\bar{A}=0$,这说明力学量$A$在任何量子态下的平均值都不会随时间改变,这是体系的一个守恒量.

\begin{theorem}[Virial定理]
    当体系处于定态时,有
    $$
        2\overline{T} = \overline{\boldsymbol{r} \cdot \nabla V}
    $$
    其中$T=\frac{p^2}{2m}$是粒子动能,$V\left(\boldsymbol{r}\right)$是势能.
\end{theorem}
\begin{proof}
    系统的Hamilton算符为
    $$
        \hat{H}=\frac{\hat{p}^2}{2m}+V(\boldsymbol{r})
    $$
    考虑$\boldsymbol{r}\cdot\boldsymbol{p}$的平均值随时间演化\footnote{不显含$t$的力学量$A$,其平均值随时间演化
        $$
            \frac{\mathrm{d}}{\mathrm{d}t}\bar{A}=\frac{1}{\mathrm{i}\hbar}\overline{\left[A, H\right]}
        $$
        如果在此基础上,力学量$A$又恰好与$H$对易,则$\frac{\mathrm{d}\bar{A}}{\mathrm{d}t}=0$,也即$A$是体系的一个守恒量。
    }
    $$
        \mathrm{i}\hbar\frac{\mathrm{d}}{\mathrm{d}t}\overline{\boldsymbol{r}\cdot\boldsymbol{p}}
        =\overline{\left[\boldsymbol{r}\cdot\boldsymbol{p}, H\right]}
        =\frac{1}{2m}\overline{\left[\boldsymbol{r}\cdot\boldsymbol{p}, \hat{p}^2\right]}+\overline{\left[\boldsymbol{r}\cdot\boldsymbol{p}, V(\boldsymbol{r})\right]}
        % =\mathrm{i}\hbar\left(\frac{1}{m}\overline{\hat{p}^2}-\overline{\boldsymbol{r}\cdot\nabla{V}}\right)
    $$
    \paragraph{对于第一项}
    因式$\left[\boldsymbol{r}\cdot\boldsymbol{p}, \hat{p}^2\right]$可以写为
    $$
        \left[\boldsymbol{r}\cdot\hat{\boldsymbol{p}}, \hat{p}^2\right]
        =\left[x\hat{p}_x + y\hat{p}_y + z\hat{p}_z, \hat{p}_x^2+\hat{p}_y^2+\hat{p}_z^2\right]
        =\left[x\hat{p}_x, \hat{p}_x^2\right] + \left[y\hat{p}_y, \hat{p}_y^2\right] + \left[z\hat{p}_z, \hat{p}_z^2\right]
    $$
    这显然具有极佳的轮换对称性,我们根据对易式的代数恒等式\footnote{
        对易式的代数恒等式:$$\left[\hat{A}\hat{B}, \hat{C}\right]=\hat{A}\left[\hat{B}, \hat{C}\right]+\left[\hat{A}, \hat{C}\right]\hat{B}$$
    }
    和量子力学的基本对易式\footnote{
        量子力学的基本对易式:$$[x, \hat{p}_x]=\mathrm{i}\hbar$$
    }处理其中一项
    $$
        \begin{aligned}
            \left[x\hat{p}_x, \hat{p}_x^2\right]
             & =x\left[\hat{p}_x, \hat{p}_x^2\right] + \left[x, \hat{p}_x^2\right]\hat{p}_x                   \\
             & =0 + \left[x, \hat{p}_x^2\right]\hat{p}_x                                                      \\
             & =0 + \left[x, \hat{p}_x\hat{p}_x\right]\hat{p}_x                                               \\
             & =0 + \hat{p}_x\left[x, \hat{p}_x\right]\hat{p}_x + \left[x, \hat{p}_x\right]\hat{p}_x^2        \\
             & = \hat{p}_x(\mathrm{i}\hbar)\hat{p}_x+(\mathrm{i}\hbar)\hat{p}_x^2=2\mathrm{i}\hbar\hat{p}_x^2
        \end{aligned}
    $$
    类似地,可以得到
    $$
        \left[\boldsymbol{r}\cdot\hat{\boldsymbol{p}}, \hat{p}^2\right] = 2\mathrm{i}\hbar\left(\hat{p}_x^2 + \hat{p}_y^2 + \hat{p}_z^2\right)= 2\mathrm{i}\hbar\hat{p}^2
    $$
    \paragraph{对于第二项} 也利用相同的对易代数恒等式展开
    $$
        \left[\boldsymbol{r}\cdot\boldsymbol{p}, V(\boldsymbol{r})\right]
        = \boldsymbol{r}\left[\boldsymbol{p}, V(\boldsymbol{r})\right] + \left[\boldsymbol{r}, V(\boldsymbol{r})\right]\boldsymbol{p}
        =\boldsymbol{r}[\boldsymbol{p}, V(\boldsymbol{r})]+0
        = -\mathrm{i}\hbar\boldsymbol{r}\nabla{V(\boldsymbol{r})}
    $$
    考虑到定态下$\frac{\mathrm{d}}{\mathrm{d}t}\overline{\boldsymbol{r}\cdot\boldsymbol{p}}=0$,所以
    $$
        2\overline{T} = \overline{\boldsymbol{r}\cdot\nabla{V(r)}}
    $$
\end{proof}


\begin{question}{题目4.2}
    设体系有两个粒子,每个粒子可处于三个单粒子态$\varphi_1$,$\varphi_2$,$\varphi_3$中的任何一个态。试求体系可能的态数目,分三种情况讨论:两个全同玻色子、两个全同费米子、两个不同粒子。
\end{question}
\begin{solution}
    \begin{itemize}
        \item 全同玻色子:8个态
        \item 全同费米子:6个态
        \item 不同粒子:9个态
    \end{itemize}
\end{solution}


\begin{question}{题目4.3}

\end{question}
\begin{solution}
    如果不考虑波函数的交换对称性,其可能的态数目为
    $$
        3^3=27
    $$
    如果要求波函数是交换反对称的,其可能的态数目为
    $$
        1
    $$
    如果要求波函数是交换对称的,其可能的态数目为
    $$
        1+6+3=10
    $$
\end{solution}






\begin{question}{题目4.4}
    设力学量 $A$ 不显含 $t$,$H$为体系的Hamilton量,证明:
    $$
        -\hbar^2\frac{\mathrm{d}^2}{\mathrm{d}t^2}\bar{A} = \overline{[[A, H], H]}.
    $$
\end{question}
\begin{solution}
    因为力学量$A$不显含$t$
    $$
        \frac{\mathrm{d}}{\mathrm{d}t}\bar{A}=\frac{1}{\mathrm{i}\hbar}\overline{[A, H]}
    $$
    上式两边再对$t$求导,则有
    $$
        \frac{\mathrm{d}^2}{\mathrm{d}t^2}\bar{A}
        =\frac{\mathrm{d}}{\mathrm{d}t}\frac{1}{\mathrm{i}\hbar}\overline{[A, H]}
        =\frac{1}{\mathrm{i}\hbar}\overline{\left[\frac{1}{\mathrm{i}\hbar}[A, H], H\right]}
        =-\frac{1}{\hbar^2}\overline{\left[[A, H], H\right]}
    $$
    简单整理得到
    $$
        -\hbar^2\frac{\mathrm{d}^2}{\mathrm{d}t^2}\bar{A} = \overline{[[A, H], H]}
    $$
    证毕.
\end{solution}




\begin{question}{题目4.5(期末考试的证明题)}
    设力学量$A$不显含$t$,证明在束缚定态下
    $$
        \frac{\mathrm{d}\bar{A}}{\mathrm{d}t} = 0
    $$
\end{question}
\begin{solution}
    定态$\psi$是体系的能量本征态,且束缚态可以归一化
    $$
        \langle\psi, \psi\rangle = \text{有限值}
    $$
    因为力学量$A$不显含$t$
    $$
        \frac{\mathrm{d}\bar{A}}{\mathrm{d}t}=\frac{1}{\mathrm{i}\hbar}\overline{[A, H]}
    $$
    所以
    $$
        \frac{\mathrm{d}\bar{A}}{\mathrm{d}t}
        =\frac{1}{\mathrm{i}\hbar}\frac{(\psi, [A,H]\psi)}{(\psi, \psi)}
        =\frac{1}{\mathrm{i}\hbar}\frac{(\psi, AH\psi)-(\psi, HA\psi)}{(\psi, \psi)}
    $$
    利用能量本征方程和Hermitian算符的性质
    $$
        \begin{aligned}
             & (\psi, AH\psi) = (\psi, AE\psi) \\
             & (\psi, HA\psi) = (H\psi, A\psi)
        \end{aligned}
    $$
    得到
    $$
        \frac{\mathrm{d}\bar{A}}{\mathrm{d}t}
        =\frac{1}{\mathrm{i}\hbar}\frac{(\psi, AE\psi)-(H\psi, A\psi)}{(\psi, \psi)}
        =\frac{E}{\mathrm{i}\hbar}\frac{(\bar{A}-\bar{A})}{(\psi, \psi)}
        =0
    $$
    证毕.
\end{solution}





