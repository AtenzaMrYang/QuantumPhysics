\section{常用数学工具}


\subsection{Fourier变换}
\begin{enumerate}
    \item 实数形式的傅里叶正弦变换
          \begin{equation}\label{实数形式的傅里叶正弦变换}
              \begin{dcases}
                  f(x)=\sqrt{\frac{2}{\pi}}\int_0^{+\infty}B(\omega)\sin{\omega x}\,\mathrm{d}x      \\
                  B(\omega)=\sqrt{\frac{2}{\pi}}\int_0^{+\infty}f(\xi)\sin{\omega\xi}\,\mathrm{d}\xi \\
              \end{dcases}
          \end{equation}
    \item 实数形式的傅里叶余弦变换
          \begin{equation}\label{实数形式的傅里叶余弦变换}
              \begin{dcases}
                  f(x)=\sqrt{\frac{2}{\pi}}\int_0^{+\infty}A(\omega)\cos{\omega x}\,\mathrm{d}x      \\
                  A(\omega)=\sqrt{\frac{2}{\pi}}\int_0^{+\infty}f(\xi)\cos{\omega\xi}\,\mathrm{d}\xi \\
              \end{dcases}
          \end{equation}
    \item 复数形式的傅里叶变换
          \begin{equation}\label{复数形式的傅里叶变换}
              \begin{dcases}
                  F(\omega)=\frac{1}{\sqrt{2\pi}}\int_{-\infty}^{+\infty}f(x)\left[\mathrm{e}^{\mathrm{i}\omega x}\right]^*\,\mathrm{d}x \\
                  f(x)=\frac{1}{\sqrt{2\pi}}\int_{-\infty}^{+\infty}F(\omega)\mathrm{e}^{\mathrm{i}\omega x}\,\mathrm{d}\omega           \\
              \end{dcases}
          \end{equation}
\end{enumerate}



\subsection{\texorpdfstring{$\delta$}{δ}函数}
虽然$\delta$函数并没有具体解析式,但有一种十分符合物理直觉的定义
$$
    \delta(x-x_0) = \begin{cases}
        \infty, & x = x_0    \\
        0,      & x \neq x_0 \\
    \end{cases} \quad
    \int_{-\infty}^{+\infty} \delta(x-x_0) \,\mathrm{d}x = 1
$$
\begin{enumerate}
    \item 原函数
          \begin{equation}\label{delta函数的原函数}
              H(x)=\int_{-\infty}^{x} \delta(t) \,\mathrm{d}t
              =\begin{cases}
                  0, & x<0 \\
                  1, & x>0 \\
              \end{cases}
          \end{equation}
    \item 奇偶性
          \begin{equation}\label{delta函数的奇偶性}
              \begin{aligned}
                  \delta(-x)  & = \delta(x)   \\
                  \delta'(-x) & = -\delta'(x) \\
              \end{aligned}
          \end{equation}
    \item 挑选性
          \begin{equation}\label{delta函数的挑选性}
              \int_{a}^{b}\delta(x-x_0)f(x)\,\mathrm{d}x
              =\begin{cases}
                  f(x_0), & x_0\in(a,b)    \\
                  0,      & x_0\notin(a,b) \\
              \end{cases}
          \end{equation}
\end{enumerate}

\subsection{Kronecker函数}
\begin{equation}
    \delta_{ij} = \begin{cases}
        0, & i \neq j \\
        1, & i = j    \\
    \end{cases}
\end{equation}

\subsection{Laplace变换}
\begin{equation}\label{拉普拉斯变换}
    \begin{dcases}
        \bar{f}(p)=\int_{0}^{+\infty}f(t)\mathrm{e}^{-pt}\,\mathrm{d}t \\
        f(t)=\frac{1}{2\pi\mathrm{i}}\int_{\sigma-\mathrm{i}\infty}^{\sigma+\mathrm{i}\infty}\bar{f}(p)\mathrm{e}^{pt}\,\mathrm{d}p
    \end{dcases}
\end{equation}


\subsection{Hermite多项式}
变系数微分方程
\begin{equation}
    u''-2zu'+(\lambda-1)u=0
\end{equation}
的系数必须满足
\begin{equation}\label{变系数微分方程有解的条件}
    \lambda-1=2n, \quad n=0,1,2\cdots
\end{equation}
其无穷级数解才会中断为一个多项式,多项式的生成函数为
\begin{equation}\label{hermite多项式的生成函数}
    \mathrm{e}^{-s^2+2zs}=\sum_{n=0}^{\infty}\frac{H_n(z)}{n!}s^n
\end{equation}
由此可以证明Hermite多项式的正交归一性
\begin{equation}\label{hermite多项式的正交归一性}
    \int_{-\infty}^{+\infty}H_m(z)H_n(z)\mathrm{e}^{-z^2}\,\mathrm{d}z=\sqrt{\pi}2^n\cdot n!\delta_{mn}
\end{equation}
和递推关系
\begin{gather}\label{hermite多项式的递推关系}
    H_{n+1}(z)-2zH_n(z)+2nH_{n-1}(z)=0  \\
    H_n'(z)=2nH_{n-1}(z)
\end{gather}

\subsection{常用积分公式}\label{常用积分公式}
\begin{equation}
    \int x\sin{ax} \,\mathrm{d}x = \frac{1}{a^2}\sin{ax}-\frac{1}{a}x\cos{ax}+C
\end{equation}
\begin{equation}
    \int x^2\sin{ax}\,\mathrm{d}x = -\frac{1}{a}x^2\cos{ax}+\frac{2}{a^2}x\sin{ax}+\frac{2}{a^3}\cos{ax}+C
\end{equation}
\begin{equation}
    \int x\cos{ax}\,\mathrm{d}x = \frac{1}{a^2}\cos{ax}+\frac{1}{a}x\sin{ax}+C
\end{equation}
\begin{equation}
    \int x^2\cos{ax}\,\mathrm{d}x = \frac{1}{a}x^2\sin{ax}+\frac{2}{a^2}x\cos{ax}-\frac{2}{a^3}\sin{ax}+C
\end{equation}


\subsection{求矩阵的逆矩阵}
求逆矩阵的相关方法大致可以分为四类
\subsubsection{定义法} 寻找一个与$\boldsymbol{A}$同阶的方阵$\boldsymbol{B}$,使得
$$
    \boldsymbol{AB}=\boldsymbol{E}
    ,\quad
    \boldsymbol{BA}=\boldsymbol{E}
$$
\subsubsection{公式法}
可利用矩阵行列式和代数余子式构成的伴随矩阵来求逆矩阵
$$
    \boldsymbol{A}^{-1} = \frac{1}{|\boldsymbol{A}|}\boldsymbol{A}^*
$$
例如矩阵$\begin{pmatrix} 1 & 2 \\ 2 & 5 \end{pmatrix}$的伴随矩阵为
$$
    \boldsymbol{A}^* = \begin{pmatrix}
        M_{11}  & -M_{12} \\
        -M_{21} & M_{22}
    \end{pmatrix} = \begin{pmatrix}
        5  & -2 \\
        -2 & 1
    \end{pmatrix}
$$
于是逆矩阵为
$$
    \boldsymbol{A}^{-1}
    =\frac{1}{|\boldsymbol{A}|}\boldsymbol{A}^*
    =\begin{pmatrix}
        5  & -2 \\
        -2 & 1
    \end{pmatrix}
$$
\subsubsection{初等变换法}
$$
    \begin{pmatrix}
        \boldsymbol{A} & \vdots & \boldsymbol{E}
    \end{pmatrix}
    \xrightarrow[]{\text{初等行变换}}
    \begin{pmatrix}
        \boldsymbol{E} & \vdots & \boldsymbol{A}^{-1}
    \end{pmatrix}
$$
$$
    \begin{pmatrix}
        \boldsymbol{A} \\
        \cdots         \\
        \boldsymbol{E}
    \end{pmatrix}
    \xrightarrow[]{\text{初等列变换}}
    \begin{pmatrix}
        \boldsymbol{E} \\
        \cdots         \\
        \boldsymbol{A}^{-1}
    \end{pmatrix}
$$
\subsubsection{分块矩阵法}
当$\boldsymbol{A}, \boldsymbol{B}$均可逆时
$$
    \begin{pmatrix}
        \boldsymbol{A} & \boldsymbol{O} \\
        \boldsymbol{O} & \boldsymbol{B}
    \end{pmatrix}^{-1}
    =\begin{pmatrix}
        \boldsymbol{A}^{-1} & \boldsymbol{O}      \\
        \boldsymbol{O}      & \boldsymbol{B}^{-1}
    \end{pmatrix}
    ,\quad
    \begin{pmatrix}
        \boldsymbol{O} & \boldsymbol{A} \\
        \boldsymbol{B} & \boldsymbol{O}
    \end{pmatrix}^{-1}
    \begin{pmatrix}
        \boldsymbol{O}      & \boldsymbol{A}^{-1} \\
        \boldsymbol{B}^{-1} & \boldsymbol{O}
    \end{pmatrix}
$$


\subsection{Virial定理}
\begin{question}{Virial定理}
    当体系处于定态时,有
    $$
        2\overline{T} = \overline{\boldsymbol{r}\cdot\nabla{V(r)}}
    $$
    其中$T=\dfrac{p^2}{2m}$是粒子动能,$V\left(\boldsymbol{r}\right)$是势能.
\end{question}
\begin{proof}
    系统的Hamilton算符为
    $$
        \hat{H}=\frac{\hat{p}^2}{2m}+V(\boldsymbol{r})
    $$
    考虑$\boldsymbol{r}\cdot\boldsymbol{p}$的平均值随时间演化\footnote{不显含$t$的力学量$A$,其平均值随时间演化
        $$
            \frac{\mathrm{d}}{\mathrm{d}t}\bar{A}=\frac{1}{\mathrm{i}\hbar}\overline{\left[A, H\right]}
        $$
    }
    $$
        \mathrm{i}\hbar\frac{\mathrm{d}}{\mathrm{d}t}(\boldsymbol{r}\cdot\boldsymbol{p})
        =\overline{\left[\boldsymbol{r}\cdot\boldsymbol{p}, H\right]}
        =\frac{1}{2m}\overline{\left[\boldsymbol{r}\cdot\boldsymbol{p}, \hat{p}^2\right]}+\overline{\left[\boldsymbol{r}\cdot\boldsymbol{p}, V(\boldsymbol{r})\right]}
        % =\mathrm{i}\hbar\left(\frac{1}{m}\overline{\hat{p}^2}-\overline{\boldsymbol{r}\cdot\nabla{V}}\right)
    $$
    \paragraph{对于第一项}
    因式$\left[\boldsymbol{r}\cdot\boldsymbol{p}, \hat{p}^2\right]$可以写为
    $$
        \left[\boldsymbol{r}\cdot\hat{\boldsymbol{p}}, \hat{p}^2\right]
        =\left[x\hat{p}_x + y\hat{p}_y + z\hat{p}_z, \hat{p}_x^2+\hat{p}_y^2+\hat{p}_z^2\right]
        =\left[x\hat{p}_x, \hat{p}_x^2\right] + \left[y\hat{p}_y, \hat{p}_y^2\right] + \left[z\hat{p}_z, \hat{p}_z^2\right]
    $$
    这显然具有极佳的轮换对称性,我们根据对易式的代数恒等式
    $$
        \left[\hat{A}\hat{B}, \hat{C}\right]=\hat{A}\left[\hat{B}, \hat{C}\right]+\left[\hat{A}, \hat{C}\right]\hat{B}
    $$
    处理其中一项
    $$
        \begin{aligned}
            \left[x\hat{p}_x, \hat{p}_x^2\right]
             & =x\left[\hat{p}_x, \hat{p}_x^2\right] + \left[x, \hat{p}_x^2\right]\hat{p}_x            \\
             & =0 + \left[x, \hat{p}_x^2\right]\hat{p}_x                                               \\
             & =0 + \left[x, \hat{p}_x\hat{p}_x\right]\hat{p}_x                                        \\
             & =0 + \hat{p}_x\left[x, \hat{p}_x\right]\hat{p}_x + \left[x, \hat{p}_x\right]\hat{p}_x^2 \\
        \end{aligned}
    $$
    再利用量子力学的基本对易式$[x, \hat{p}_x]=\mathrm{i}\hbar$,得到
    $$
        \left[x\hat{p}_x, \hat{p}_x^2\right]
        =\hat{p}_x(\mathrm{i}\hbar)\hat{p}_x + (\mathrm{i}\hbar)\hat{p}_x^2
        =2\mathrm{i}\hbar\hat{p}_x^2
    $$
    同理可得
    $$
        \left[y\hat{p}_y, \hat{p}_y^2\right] = 2\mathrm{i}\hbar\hat{p}_y^2 \quad
        \left[z\hat{p}_z, \hat{p}_z^2\right] = 2\mathrm{i}\hbar\hat{p}_z^2
    $$
    于是
    $$
        \left[\boldsymbol{r}\cdot\hat{\boldsymbol{p}}, \hat{p}^2\right] = 2\mathrm{i}\hbar\left(\hat{p}_x^2 + \hat{p}_y^2 + \hat{p}_z^2\right)
        =2\mathrm{i}\hbar\hat{p}^2
    $$
    \paragraph{对于第二项} 也利用相同的对易代数恒等式展开
    $$
        \left[\boldsymbol{r}\cdot\boldsymbol{p}, V(\boldsymbol{r})\right]
        = \boldsymbol{r}\left[\boldsymbol{p}, V(\boldsymbol{r})\right] + \left[\boldsymbol{r}, V(\boldsymbol{r})\right]\boldsymbol{p}
    $$
    显然$\boldsymbol{r}$和$V(\boldsymbol{r})$相互对易
    $$
        \left[\boldsymbol{r}, V(\boldsymbol{r})\right]=0
    $$
    再将算符$\boldsymbol{r}\left[\boldsymbol{p}, V(\boldsymbol{r})\right]$作用在任意波函数$\psi$上
    $$
        \begin{aligned}
            \boldsymbol{r}\left[\boldsymbol{p}, V(\boldsymbol{r})\right]\psi
             & =\hat{p}[V(\boldsymbol{r})\psi]-V(\boldsymbol{r})\hat{p}\psi                                                                                \\
             & = -\mathrm{i}\hbar\nabla[V(\boldsymbol{r})\psi] + \mathrm{i}\hbar V(\boldsymbol{r})\nabla\psi                                               \\
             & = -\mathrm{i}\hbar V(\boldsymbol{r})\nabla\psi - \mathrm{i}\hbar\psi\nabla[V(\boldsymbol{r})] + \mathrm{i}\hbar V(\boldsymbol{r})\nabla\psi \\
             & = -\mathrm{i}\hbar\psi[\nabla V(\boldsymbol{r})]
        \end{aligned}
    $$
    所以这个算符的本质是
    $$
        \left[\hat{\boldsymbol{p}}, V(\boldsymbol{r})\right] = -\mathrm{i}\hbar\nabla{V(\boldsymbol{r})}
    $$
\end{proof}




